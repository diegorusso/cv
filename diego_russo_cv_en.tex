%% Copyright 2006-2010 Xavier Danaux (xdanaux@gmail.com).
%
% This work may be distributed and/or modified under the
% conditions of the LaTeX Project Public License version 1.3c,
% available at http://www.latex-project.org/lppl/.

\documentclass[10pt,a4paper]{moderncv}

% moderncv themes
\moderncvtheme[darkred]{classic}                 % optional argument are 'blue' (default), 'orange', 'red', 'green', 'grey' and 'roman' (for roman fonts, instead of sans serif fonts).

% character encoding
\usepackage[utf8]{inputenc}                   % replace by the encoding you are using
\usepackage{footmisc} % enabling footnotes.

% adjust the page margins
\usepackage[scale=0.85]{geometry}
\setlength{\hintscolumnwidth}{2.5cm}              % if you want to change the width of the column with the dates
% \AtBeginDocument{\setlength{\maketitlenamewidth}{6cm}}  % only for the classic theme, if you want to change the width of your name placeholder (to leave more space for your address details
\AtBeginDocument{\recomputelengths}                     % required when changes are made to page layout lengths

% personal data
\firstname{Diego}
\familyname{Russo}
\title{Software Engineer}               % optional, remove the line if not wanted
\address{Cambridge}{United Kingdom}    % optional, remove the line if not wanted
\mobile{+44 7428 251191}                    % optional, remove the line if not wanted
\email{me@diegor.it}                      % optional, remove the line if not wanted
\homepage{http://www.diegor.co.uk}                % optional, remove the line if not wanted
\extrainfo{last update: May 2017} % optional, remove the line if not wanted.
\photo[100pt]{images/diegor}                         % '64pt' is the height the picture must be resized to and 'picture' is the name of the picture file; optional, remove the line if not wanted
\quote{"Now is better than never" - The Zen of Python}                 % optional, remove the line if not wanted

% to show numerical labels in the bibliography; only useful if you make citations in your resume
\makeatletter
\renewcommand*{\bibliographyitemlabel}{\@biblabel{\arabic{enumiv}}}
\makeatother

% bibliography with mutiple entries
%\usepackage{multibib}
%\newcites{book,misc}{{Books},{Others}}

%\nopagenumbers{}                             % uncomment to suppress automatic page numbering for CVs longer than one page

\begin{document}
\maketitle

\section{Desired employment}
\cvline{}{\large\textbf{I'm always looking for a challenging position where I can express and use my passion for programming and technology. I mostly develop in Python in *NIX environment on a daily basis. As I'm a dynamic person, professional and personal growth are very important to me.}}

\section{Experience}
\subsection{Vocational}
\cventry{\textbf{2016/06--Current Position}}{Staff Software Engineer in DSG (Development Solutions Group)}{ARM Ltd, \url{http://www.arm.com/}}{Cambridge, UK}{}{I am the lead of a combined infrastructure team (8) looking after both the ARM Compiler and OSS Toolchains infrastructures (GNU-A, GNU-RM and LLVM OSS). Besides the technical lead, my role is to make sure all infrastructures respect a set of design principles whilst sharing knowledge, technologies and processes across different teams. I keep a monthly forum with stakeholders to be sure to catch their requirements and I represent DSG infrastructures (not only compilers) across the business.}

\cventry{2013/11--2016/05}{Staff Software Engineer in DSG (Development Solutions Group)}{ARM Ltd, \url{http://www.arm.com/}}{Cambridge, UK}{}{Working in the \textbf{GNU} team, our goal is the delivery of the \textbf{GNU toolchain} \textit{(ld, newlib, binutils, gas, gcc, libffi, g++, gfortran, libgomp, libstdc++)} optimized for ARM processors. I am responsible of the infrastructure for automatic builds, tests and benchmarks. Most of the infrastructure is \textbf{Python} based and it automates all the phases of the process: checkout of the code (using GIT), build, systematic tests (DejaGnu) through LSF (Load Sharing Facility) and the local aarch64 board farm, and performance tests. I also developed an application that collects test's outcomes and it stores them into a database (\textbf{MongoDB}) in order to analyse those data and to find error root causes, trends, common patterns and automating the tracking of issues using JIRA. For this task I am using \textbf{Flask} for APIs, \textbf{Bootstrap} and \textbf{AngularJS} for the frontend side.}

\cventry{2011/10--2013/10}{Senior Software Developer in Engineering IT Department}{ARM Ltd, \url{http://www.arm.com/}}{Cambridge, UK}{}{Working in a team, I'm involved in many internal projects using \textbf{CentOS} and mainly the following languages: \textbf{Python, Java, Perl, C, tcsh and bash}. I developed from scratch a reliable and fault tolerant application that interacts with the cluster (\textbf{LSF}) and a AMQ server (\textbf{RabbitMQ}). For this project the main language has been Python using a NOSQL database (\textbf{MongoDB} configured as \textit{ReplicaSet}). I've also developed a JIRA plugin to interact with an internal software in order to synchronise external tickets with internal ones. I look after, improve and fix many IT internal software using a wide range of languages. I've got experience also with LSF cluster, customizing deeply its behaviour in order to provide a functional solution to our customer. Other minor projects are related to \textbf{SVN hooks}, FlexNet Manager server, LSF monitor, internal application interacting with distributed storage. With a colleague of mine, I'm running the IT ECS (\textit{Early Career Scheme}), managing all the phases from the CV sifting to the early career in ARM of graduates and interns.}

\cventry{2006/12--2008/08 2009/09--2011/09}{Python/Django Programmer}{Consorzio Miles - Servizi Integrati, Rome}{Assisi, IT}{}{Working in a team, I developed a management application for the municipality of Bettona using Django, Python, PostgreSQL, Linux, Apache, for the \textbf{computerization of services}, the management of personal data, building practices, urban planning, calculation of ICI tax and updating of land registry data. Also I created an advanced web interface for sending proposed practices, on-line services conference, integration process, exploration of cadastral map in \textbf{DXF} and production of customized automated printing. During the project I used revision control systems (SVN/GIT) with related web interface (trac) to manage tickets.}

\cventry{2011/05--2011/06}{Objective-C Programmer}{Forinicom Srl, Bastia Umbra, \url{http://www.com-com.it}}{Assisi, IT}{}{Developed an iPhone application that permit you to auto-login into ComCom Hotspot. This application is used by the attendees in the Europython Conference 2011 in Florence.}

\cventry{2008/04--2011/02}{Programmer and System Engineer in Research and Development Department}{Forinicom Srl, \url{http://www.com-com.it}}{Bastia Umbra, IT}{}{Working in a research and development team to create an innovative and unique product in the wireless communications market (WiFi). I initially worked on \textbf{embedded systems} (ubnt, alix, pcengines), customizing the operating system (ubnt, openwrt) and the software to manage authentication (hostapd, wpa-supplicant). After this first phase, I focused on the software \textbf{to flash} these devices and on large-scale production software. Also, we developed a complete solution for managing \textbf{hotspots}: I worked on server-side development to manage authentication, sessions log, signups, signals management from remote devices, integration with our management software, payment via credit card and authentication via SMS, complying with Pisanu law. My final task was to create software for network monitoring. It is a \textbf{PyQT stand-alone} application, using internal django based API. The technologies mostly used are Python/Django with PostgreSQL database on Debian OS virtualized on XEN}

\cventry{2010/11--2011/01}{Python/Django Programmer}{Exion Sagl, Manno, Switzerland, \url{http://www.exion.ch/}}{Remote, Assisi, IT}{}{Setting up an \textbf{Adult WebTV} entirely developed in Python/Django with PostgreSQL database on Linux/Apache platform and Red5 as streaming server. The work is managed independently using GIT as revision control system.}

\cventry{2010/10--2011/01}{Python/Pylons Programmer}{Sauce Labs Inc, San Francisco, California, USA, \url{http://saucelabs.com/}}{Remote, Assisi, IT}{}{Implementing new features, bug fixing, structural changes to the site of Sauce Labs. Distance work coordination. Site is developed in Python/Pylons using \url{github.com} as revision control system platform.}

\subsection{Miscellaneous}
\cventry{2013/06--2015/06}{Staff Software Engineer}{Opentaste Ltd \url{http://www.opentaste.eu}}{Remote, Cambridge, UK}{}{Working part-time, I am part of a distributed team of 15-20 people across the globe (San Francisco, Italy, Australia). My main tasks are technical counselor and code reviewer. Opentaste is entirely written in \textbf{Python} using \textbf{Flask} as web framework and \textbf{MongoDB} as database. Communication is crucial and we do organise regular hangouts to discuss about plans, issues and tasks. We are using \textbf{github} and \textbf{Google Docs} for managing our work }

\cventry{2016/01--2016/06}{iPhone (iOS9) and Swift 2.2}{University of Perugia, Computer Science department, \url{http://informatica.unipg.it}}{Online, Udemy - Cambridge, UK}{}{After self learning Swift 2.2, I've attended an online course to create iOS apps. I've used those skills to create DipMatBeacon (\url{https://github.com/diegorusso/DipMatBeacon}), an app used to check the booking state of Maths department rooms using iBeacon technology. Patterns used to write the app: MVC, delegation, protocols, safe programming patterns. Other features: search, share, TouchID, Reachability, iBeacon integration.}

\cventry{2011/06}{Teaching - Advanced computer course}{Centro Studi Citt\'a di Foligno, \url{http://www.cstudifoligno.it/}}{Foligno, IT}{}{Taught a class of 10 people the existence of the open source world, installing open source software on Windows and then proceed to install Ubuntu on their laptop.}

\cventry{2011/01--2011/06}{iPhone (iOS4) and Objective-C}{University of Perugia, Computer Science department, \url{http://informatica.unipg.it}}{Assisi, IT}{}{Following lessons of Stanford University, I trained myself to Objective-C and iPhone world, developing small applications. As final project I customized a VOIP application for iPhone based on Linphone (\url{http://www.linphone.org/}).}

\cventry{2005/11--2006/05}{Trainee - S.E.O. Search Engine Optimization}{WEDOIT sas, \url{http://www.wedoit.us}}{Assisi, IT}{}{Working in a team I acquired knowledge of S.E.O. and its behavior. The internship included S.E.O. optimization of various websites, using \emph{pagerank} and \emph{link popularity} methods. Also I worked as a system engineer of Debian-based virtualised server and I developed a S.E.O. oriented application in Python and PHP.}

\cventry{2002/02}{Trainee combined with IFS project, Impresa Formativa Simulata (Enterprise Training Simulation)}{IOSA CARLO Srl, \url{http://www.iosacarlo.com}}{Terni, IT}{}{Administration of enterprise network}

\section{Education and Trainings}
\cventry{\textbf{Since 2010/10}}{Specialization course, with mayor in Computer Science, ``Security''}{University of Perugia, Computer Science department, \url{http://informatica.unipg.it}}{Perugia, IT}{\textit{Enrolled}}{Exames passed with excellent marks: parallel computing, simulation, advanced programming and lab, advanced operating systems and lab, theoretical computer science, computer security, advanced databases and data mining, law applied to computer science and communications, cryptography, mobile programming.}

\cventry{2015/10}{MongoDB Administration and Developers training}{MongoDB}{ARM Ltd - Cambridge, UK}{}{3 days training covering MongoDB query language, data modeling, indexes, aggregation framework, replica set, sharding, backup, restore and basic administration.}

\cventry{2015/04}{Puppet Fundamentals}{Puppet Labs}{ARM Ltd - Cambridge, UK}{}{3 days training covering Puppet fundamentals: \url{http://bit.ly/puppet-fundamentals}}

\cventry{2014/01--2014/03}{Cryptography 1}{Coursera}{Online - Cambridge, UK}{Final score: \textbf{PASSED}}{Topics covered in the course: semantic security, block ciphers and pseudorandom functions, DES/AES block ciphers, message integrity, collision resistant hashing, authenticated encryption, Diffie-Hellman, RSA, and Merkle puzzles, public key encryption, the ElGamal system (\url{http://bit.ly/coursera-crypto1})}

\cventry{2013/09}{ITIL Foundation}{ILX Group}{ARM Ltd, Cambridge, UK}{Foundation Score: \textbf{31 out of 40, PASSED}}{The skills achieved are all included in (\url{http://bit.ly/itil-foundation})}

\cventry{2013/10--2014/01}{Brazilian Portuguese Course}{Private Classes}{Cambridge, UK}{\textbf{B1 level}}{The skills achieved are all included in \textbf{Common European Framework of Reference for Languages}\footnotemark[1]}

\cventry{2011/06 2012/07 2013/07 2014/06 2015/06 2016/06 2017/07}{Europython Conference}{Volunteer, Attendee}{Florence, Italy/Berlin Germany, Bilbao/Spain, Rimini/Italy}{}{I volunteered in the 2011 edition and attended 2012, 2014, 2015, 2016 and 2017 editions. Europython represents my annual appointment with Python world as it is source ideas, talks, trainings that keep me updated with the latest technologies around Python which I can apply to my day to day job.}

\cventry{2013/04--2013/06}{Advanced English Course}{Sixth Form College}{Cambridge, UK}{\textbf{CEF level C1-C2}}{The skills achieved are all included in \textbf{Common European Framework of Reference for Languages}\footnotemark[1]}

\cventry{2012/05--2012/06 2012/10--2012/11 2013/01--2013/02}{Brazilian Portuguese Course}{Sixth Form College}{Cambridge, UK}{\textbf{A2 level}}{The skills achieved are all included in \textbf{Common European Framework of Reference for Languages}\footnotemark[1]}

\cventry{2010/10--2011/05}{English Course}{Comprehensive School ``Volumnio'' Ponte San Giovanni}{Perugia, IT}{\textbf{B1 level}}{The skills achieved are all included in \textbf{Common European Framework of Reference for Languages}\footnotemark[1]}

\cventry{2009/10--2010/05}{Spanish Course}{Comprehensive School ``Volumnio'' Ponte San Giovanni}{Perugia, IT}{\textbf{B1 level}}{The skills achieved are all included in \textbf{Common European Framework of Reference for Languages}\footnotemark[1]}

\cventry{2009/08--2010/03}{Paper publication \cite{aes}}{University of Perugia, Computer Science department, \url{http://informatica.unipg.it}}{Perugia, IT}{}{Preparation and publication of ``The AES implementation based on OpenCL for multi/many core architecture'' paper for the yearly conference ICCSA 2010 (\url{www.iccsa.org}) at Sangyo University, Fukuoka in Japan. The paper discussed the implementation of an AES algorithm that runs on NVIDIA/ATI graphics card.}

\cventry{2007/02--2007/07}{Radio-amateur \textbf{class A license}}{C.I.S.A.R. Foligno's section}{Assisi, IT}{PASSED, International Callsign \textbf{IZ0OVB}}{During the course for radio-amateur I acquired excellent knowledge of radio technology basics, radio devices and its usage and basics of Physics and Chemistry (magnetism, elettromagnetism)}

\cventry{2007/03}{Spanish Course}{Inhispania Intlance S.L, \url{http://www.inhispania.com}}{Madrid, ES}{\textbf{A2 Level}}{During the time in Madrid, I studied Spanish grammar and general Spanish culture.}

\cventry{2006/12}{ISO certifications course}{WEDOIT sas, \url{http://www.wedoit.us}}{Assisi, IT}{}{Training course on security and ISO certifications, covering ISO 27001:2005, policy for Information Security, Risks Analysis (RA), analysis of controls of ISO 17799:2005, Risk Transfer Process (RTP), certification process, overview of existing certification audits, audit plan and checklist, audit report, a look at future certifications.}

\cventry{2002/10--2006/11}{Bachelor Degree in Computer Science}{University of Perugia, Computer Science department, \url{http://informatica.unipg.it}}{Perugia, IT}{\textbf{102/110}}{Computer science Bachelor Degree, \textbf{mayor ``Network''}: Mathematics (analytical and discrete), Programming (C, Java, Php, html, xml, xsl, dtd, Pascal, scripting bash and csh, VB.NET, VRML), Databases (Mysql, MS Access and related programming language), Networks (ATM, xDSL, Mpls, X.25, Frame Relay) types (wireless, wired) and interaction between them, Knowledge of multimedia system, overview of parallel computing (mpi)}

\cventry{1996/09--2002/06}{Accountant programmer Diploma (Mercurio project)}{Ministry of Public Education - I.T.C. (Commercial technical institute) ``Federico Cesi''}{Terni, IT}{\textbf{85/100}}{Topics covered by the course (Commercial technical institute) of study as defined by the Ministry of Public Education: Chemistry / Physics, Mathematics and Computer Laboratory, Natural Science, Word processing and data, Second foreign language (French), Law and Economics, Business, Economics and Financial Science, Italian Language and Literature, History, Computer Management, Applied Mathematics, First foreign language (English), Law.}

\cventry{2001--2002}{National Project IFS (Enterprise Training Simulation}{Ministry of Public Education - I.T.C. (Commercial technical institute) ``Federico Cesi''}{Terni, IT}{Certificate of attendance}{Simulation of waste disposal firm, backed by Iosa Carlo S.r.l. (\url{http://www.iosacarlo.com}). Within the project I coordinated the work of all students, building the simulated organization chart and programming the website.}

\footnotetext[1]{\url{http://en.wikipedia.org/wiki/Common_European_Framework_of_Reference_for_Languages}}

\section{Bachelor Thesis}
\cvline{title}{\emph{Wireless Broadband Network - Weconnect project} (2006/07--2006/12)}
\cvline{supervisors}{Simone Brunozzi, Sergio Tasso}
\cvline{description}{\small The thesis was to develop a WiFi network in order to coverage \textbf{digital-divide} areas. Thanks to this project, I acquired a wide knowledge about WiFi networks and their behavior, legislation that governs the operation, RouterOS operating system (\url{www.mikrotik.com}), AAA protocol and Radius server. Finally I administered a server for the provision of various network services: mail (Postfix), web server (Apache), DNS (pdns), firewall (iptables), database (PostgreSQL), hotspot (Chillispot), Debian OS, Voyage (OS for embedded Debian based system).}

\section{Languages}
\cvlanguage{Italian}{\textbf{Mother tongue}}{}
\cvlanguage{English}{\textbf{C1 level}}{\textbf{Preliminary English Test} (PET), 2011/05}
\cvlanguage{Spanish}{\textbf{C1 level}}{\textbf{Diploma de Español como Lengua Extranjera} (D.E.L.E.), 2010/05}
\cvlanguage{Portuguese (BR)}{\textbf{B1 level}}{}

\section{Computer skills}
\cvline{Programming, Scripting, Markup Languages}{ {\huge Python}, {\normalsize Javascript}, {\normalsize CSS}, {\large bash}, {\large HTML}, {\tiny Perl}, {\tiny XML}, {\large SQL}, {\large JSON}, {\small LSL (Linden Scripting Language)}, {\huge Java}, {\normalsize C}, {\normalsize Objective-C}, {\small PHP}, {\normalsize LaTeX}, {\normalsize Swift}}
\cvline{Frameworks}{Django, Flask, AngularJS, Bootstrap, CherryPy, JQuery, Nokia Qt4, Pylons}
\cvline{Operating Systems}{Linux (Debian based, CentOS), Unix, OSX, XEN and Virtualization, OpenWRT, Ubnt (\url{http://www.ubnt.com/}), Microsoft Windows}
\cvline{IDEs}{Vim (I use it for everything!), Eclipse, XCode}
\cvline{Databases}{MongoDB, PostgrSQL, Redis, MySQL, SQLite, CouchDB}

\section{Interests}
\cvline{Languages}{\small I've learned Spanish by myself. At the moment I speak English, Spanish, Italian and some Portuguese.}
\cvline{Technology}{\small Attracted by all things that have a processor}
\cvline{Photos}{\small Amateur photos, I own a reflex camera}
\cvline{Music}{\small Hobby level. I played piano and guitar and I love to listen to any genre, from salsa to metal}
\cvline{Studies}{\small Willing to learn and study}
\cvline{Science}{\small Attraction for science in general}
\cvline{Curious}{\small That's how I define myself}
\cvline{Sports}{\small Cuban Salsa dancing, Taichi. Past: Squash, Capoeira, Kungfu, Swimming}
\cvline{Puzzles}{\small I love any kind of puzzle and I'm very passionate about Rubik's cubes: size solved 2x2x2, 3x3x3, 4x4x4, 5x5x5, 6x6x6 and 9x9x9}

\section{Licence(s)}
\cvlistitem{Full UK Driving licence}
\cvlistitem{Operator license for amateur radio station class A (nr. 020122/AN), International callsign \textbf{IZ0OVB}}

\section{Extra Information}
\cvlistitem{Comply with compulsory military service (referring to studies)}
\cvlistitem{Linux Registered User \#399008}
\cvlistitem{Marital status: single}

% Publications from a BibTeX file without multibib\renewcommand*{\bibliographyitemlabel}{\@biblabel{\arabic{enumiv}}}% for BibTeX numerical labels
\nocite{*}
\bibliographystyle{plain}
\bibliography{diego_russo_publications}       % 'diego_russo_publications' is the name of a BibTeX file

\section{About this CV}
\input{revision}
\cvline{Language}{\LaTeX}
\cvline{Source}{\url{https://github.com/diegorusso/cv}}
\cvline{Commit}{\Revision}

\section{About me}
\cvline{}{Because of \textbf{my passion for computing}, I have developed over the years a wide range of skills many areas computer-related.\newline
During the years of higher education, beyond academic success, I created and maintained an \textbf{above average} extra-curricular activity: among other initiatives in which I participated, I remember ''The basic course on multimedia'',''Exposcuola 2000 in Paestum'', ''Basic computer as tutor role'', ''Computer literacy course as tutor to senior people (over 65 years old)'', ''XI week of scientific and technological culture'', ''Pluto Meeting 2001'' and ''Main tutor of a group of 6 people for the course of IMPROVING ITALIAN for new students in the project ''Reception, Recovery, Empowerment in early grades''''. In all the projects mentioned, I participated actively devoting time and effort in learning new things about new information technology and other subjects.\newline
From the moment I had my first computer, I have had a certain passion for the \textbf{open source} world and all that concerns it: in fact I have managed machines with \textbf{Linux} distributions such as RedHat 7.3, Slackware 7.1 up to machines to Debian (from version 3.0 to current ones). Through this experience I have gained skill and knowledge in the management of Linux: bash scripting, configuring and compiling the kernel, network services, patching the kernel, C language. In addition to Linux, I daily use OSX.\newline
I was active as a contributor to the writing of the guide \url{http://www.ubuntusemplice.org/} (version 6.06 and 7.10). On this project I was the reviewer and author of several chapters, I administered the machine that hosts the site, wiki, blog and mailing list.\newline
\textbf{I also have a great passion for programming languages} knowing many of them. I studied LSL during my work on \textbf{Second Life}: in fact I have worked on many Italian metaverse project such as Assis \url{http://www.secundavita.it}, Milan and Marostica on the \textbf{''Italia Vera''} project.\newline
I have a good knowledge of graphic applications (Gimp, Photoshop) and office tools such as Openoffice.org and iWork (for OSX)\newline
Determined, I can work both in a team and individually and I am able to independently manage my workflow. Working usually in a team, I have a constructive and collaborative relationships with the people around me. I am sociable, friendly and with good communication skills; my site is a source of continuing social contact and exchange with other technical and less technical people.}
\end{document}
