\documentclass[totpages,helvetica,openbib,english]{europecv}
\usepackage[T1]{fontenc}
\usepackage{graphicx}
\usepackage[a4paper,top=1.27cm,left=1cm,right=1cm,bottom=2cm]{geometry}
\usepackage[english]{babel}
\usepackage{bibentry}
\usepackage{url}
\usepackage{enumitem}
\setlist{nolistsep}

\renewcommand{\normalsize}{\fontsize{2.9mm}{3.1mm}\selectfont}

\ecvname{Diego Russo}
\ecvaddress{Via G. Garibaldi 40, 05021, Acquasparta (TR), Italia}
\ecvtelephone[+39 334 5873886]{+39 0744 930614}
\ecvemail{\url{diegor.it@gmail.com} - Personal, gtalk, MSN \\& \url{diego.russo@forinicom.it} - Forinicom S.r.l.}
\ecvhomepage{\url{http://www.diegor.it}}
\ecvnationality{Italian}
\ecvdateofbirth{30 April 1983}
\ecvgender{Male}
\ecvbeforepicture{\raggedleft}
\ecvpicture[width=3cm]{diegor.jpg}
\ecvafterpicture{\ecvspace{-3cm}} 
\ecvfootnote{For more information go to: \url{http://europass.cedefop.eu.int}\\
\textcopyright~European Communities, 2003.}

\begin{document}
    \begin{center}
        \hspace{1pt}
        \vspace{2cm}
    
        {\scshape \textbf{\Huge Diego Russo}}
    
        \vspace{1cm}
    
        {\scshape \textbf{\large \underline{Curriculum Vitae}}}
    
        \vspace{0.25cm}
    
        updated \emph{\textbf{20 March 2010}}
        
        \vspace{2cm}
        
        \begin{figure}[htbp] 
            \begin{center} 
                \includegraphics[width=10cm]{io.jpg}
            \end{center} 
        \end{figure}
        
    \end{center}
\pagebreak
\selectlanguage{english}

\begin{europecv}
\ecvpersonalinfo[5pt]

\ecvitem{\large\textbf{Desired employment / Occupational field}}{\large\textbf{Python and Django programmer. Linux and OSX systems engineer. Also open to other employment with opportunities for professional and personal growth.}}

\ecvsection{Work experience}

\ecvitem{Dates}{\textbf{Since 26 April 2008}}
\ecvitem{Occupation or position held}{Programmer, systems engineer (national contract part-time, 25h)}
\ecvitem{Main activities and responsibilities}{Development of a project that aims to provide connectivity and integrated services (video surveillance, VoIP) to public administrations, businesses and individuals. Points provided by the contract
\begin{itemize}
    \item know the products and services in the industry and the business environment: hierarchical and Mesh networks, VoIP services and video surveillance, security of wired and wireless networks;
    \item know and be able to apply the technical fundamentals of professionalism;
    \item know and be able to use techniques and working methods with particular reference to software development and documentation of code;
    \item know and be able to use tools and technologies of work (equipment, machinery and tools), in particular development environment Linux development platform Linux / OSX, programming languages Python and PostgreSQL / MySQL;
    \item know and use measures of individual security and environmental protection;
    \item know about innovations in product, process, context and sector.
\end{itemize}
I also developed applications in python and python-django both the daily work of the technical department and was of pure research and development. In that sense I have implemented from scratch a server to manage the \textbf{hotspot service}}
\ecvitem{Name and address of employer}{Forinicom srl, Via del Popolo, 9 Bastia Umbra, 06083, +390758001868, \url{http://www.forinicom.it}}
\ecvitem[10pt]{Type of business or sector}{Telecommunications}

\ecvitem{Dates}{\textbf{Since 03 September 2009 }}
\ecvitem{Occupation or position held}{Programmer (worker with a project contract, part-time)}
\ecvitem{Main activities and responsibilities}{Development of application management for the municipality of Bettona in django, python, postgresql, linux, apache, for the automatization of management services of master data, building practices, urban planning, calculation of ICI tax and updates cadastral data. Using a version control server (GIT). Also at this stage there was the completion of the product with its marketing.}
\ecvitem{Name and address of employer}{Consorzio Miles - Servizi Integrati, CF 04881101002, Via Rocca di Papa 21, Roma}
\ecvitem[10pt]{Type of business or sector}{Integrated services for public administration}

\ecvitem{Dates}{\textbf{From 11 Dicember 2006 to 31 August 2008}}
\ecvitem{Occupation or position held}{Programmer (worker with a project contract, full-time)}
\ecvitem{Main activities and responsibilities}{Development of application management for the municipality of Bettona in django, python, postgresql, linux, apache, for the automatization of management services of master data, building practices, urban planning, calculation of ICI tax and updates cadastral data. Using a version control server (SVN) with trac and ticket management}
\ecvitem{Name and address of employer}{Consorzio Miles - Servizi Integrati, CF 04881101002, Via Rocca di Papa 21, Roma}
\ecvitem[10pt]{Type of business or sector}{Integrated services for public administration}

\ecvitem{Dates}{\textbf{From 30 July 2006 to 30 Dicember 2006}}
\ecvitem{Occupation or position held}{Undergraduate student (Wireless Broadband Network)}
\ecvitem{Main activities and responsibilities}{Wireless Broadband Network - project \textbf{WeConnect} - Broadband over wifi network:
    \begin{itemize}
        \item Extensive knowledge of the wireless network and its behavior
        \item Good knowledge of laws on the Wi-Fi
        \item Excellent knowledge of the RouterOS system (www.mikrotik.com)
        \item Knowledge of the protocol AAA and the FreeRADIUS server
        \item Administration of various network services: mail (postfix), web server (apache), DNS (pdns), firewall (iptables), database (postgresql), hotspot (chillispot), OS Debian, Voyage (OS for embedded system, Debian based)
    \end{itemize}}
\ecvitem{Name and address of employer}{WEDOIT s.a.s. - Via Protomartiri Francescani,26 - 06088 Assisi (PG)}
\ecvitem[10pt]{Type of business or sector}{Computer science solutions}

\ecvitem{Dates}{\textbf{From 14 November 2005 to 30 May 2006}}
\ecvitem{Occupation or position held}{Trainee (S.E.O. Search Engine Optimization)}
\ecvitem{Main activities and responsibilities}{\vspace{-2mm}
    \begin{itemize}
        \item Basic SEO knowledge. SEO site optimization.
        \item Pagerank and link popularity technics.
        \item System administration of a virtual server, Debian based
        \item Python learning with implementation of some applications SEO oriented.
        \item PHP learning with implementation of some applications SEO oriented
    \end{itemize}}
\ecvitem{Name and address of employer}{WEDOIT s.a.s. - Via Protomartiri Francescani,26 - 06088 Assisi (PG)}
\ecvitem[10pt]{Type of business or sector}{Computer science solutions}

\ecvitem{Dates}{\textbf{March 2002}}
\ecvitem{Occupation or position held}{Trainee (Internship project combined with IFS, Impresa Formativa Simulata - Enterprise Training Simulation)}
\ecvitem{Main activities and responsibilities}{Administration of enterprise network}
\ecvitem{Name and address of employer}{IOSA CARLO S.r.l. - 05100 TERNI - Via Pallotta n. 7 - Tel. (0744) 2460 - Fax (0744) 246035 - P.IVA 00072550551 - \url{http://www.iosacarlo.com} - \url{iosacarlo@iosacarlo.com}}
\ecvitem[10pt]{Type of business or sector}{Waste disposal firm}

\ecvsection{Education and training}

\ecvitem{Dates}{\textbf{Since October 2008}}
\ecvitem{Title of qualification awarded}{Enrolled to specialization of Computer Science, ``Security'' address}
\ecvitem{Principal subjects/occupational skills covered}{Passed following exams with vote:\begin{itemize}
    \item Simulation: 30 with honours
    \item Advanced programming: 30 with honours
    \item Advanced Operating Systems: 30 with honours
    \item Advanced Operating Systems Laboratory: 30 with honours
\end{itemize}}
\ecvitem{Name and type of organisation providing education and training}{University of Perugia, Computer Science department}
\ecvitem[10pt]{Level in national or international classification}{-}

\ecvitem{Dates}{\textbf{Since 17 February 2010}}
\ecvitem{Title of qualification awarded}{Enrolled to 4$^\circ$ module of spanish language}
\ecvitem{Principal subjects/occupational skills covered}{The levels achieved in this form are as follows:
\begin{itemize}
    \item understand the main points of clear standard speech on familiar matters
    \item able to describe experiences and events, reasons and projects
    \item deal with most situations likely to arise whilst traveling in an area where the language is spoken
\end{itemize}}
\ecvitem{Name and type of organisation providing education and training}{Comprehensive School ``Volumnio'' Ponte San Giovanni - Perugia}
\ecvitem[10pt]{Level in national or international classification}{-}

\ecvitem{Dates}{\textbf{From August 2009 to March 2010}}
\ecvitem{Title of qualification awarded}{Paper pubblication \textbf{``The AES implentation based on OpenCL for multi/many core architecture''}}
\ecvitem{Principal subjects/occupational skills covered}{Preparing and publication of ``The AES implentation based on OpenCL for multi/many core architecture'' paper for the yaerly conference ICCSA 2010 (\url{www.iccsa.org}) at Sangyo University, Fukuoka in Japan. The paper discuss of an implementation of AES algorithm that runs on NVIDIA/ATI graphics card.}
\ecvitem{Name and type of organisation providing education and training}{University of Perugia, Computer Science department}
\ecvitem[10pt]{Level in national or international classification}{-}

\ecvitem{Dates}{\textbf{From 14 October 2009 to 12 February 2010}}
\ecvitem{Title of qualification awarded}{Certificate of attendance 42 hours in 50 of the 3$^\circ$ module Spanish language}
\ecvitem{Principal subjects/occupational skills covered}{The reached levels in this module are:
\begin{itemize}
    \item understand sentences and frequently used expressions related to areas of most immediate relevance
    \item can describe in simple terms aspects of their history and own experiences
    \item can speak of the surrounding environment and be able to express needs, intentions and predictions
\end{itemize}}
\ecvitem{Name and type of organisation providing education and training}{Comprehensive School ``Volumnio'' Ponte San Giovanni - Perugia}
\ecvitem[10pt]{Level in national or international classification}{-}

\ecvitem{Dates}{\textbf{From August 2007 to August 2008}}
\ecvitem{Title of qualification awarded}{Book publishing \textbf{UbuntuSemplice 7.10} with donation to Canonical Ltd.}
\ecvitem{Principal subjects/occupational skills covered}{\vspace{-2mm}
\begin{itemize}
    \item Contributor, author of many chapters and system engineer of \url{http://www.ubuntusemplice.org/}
    \item Debian virtual server administration
    \item Configuring and using the MediaWiki collaborative software for drawing and composition of the book
    \item Web server Apache configuration
    \item Mailing list configuration to manage the collaborative writing of the book
\end{itemize}}
\ecvitem{Name and type of organisation providing education and training}{Ubuntusemplice Project - \url{http://www.ubuntusemplice.org/)}}
\ecvitem[10pt]{Level in national or international classification}{Excellent knowledge of Ubuntu}

\ecvitem{Dates}{\textbf{From February 2007 to July 2007}}
\ecvitem{Title of qualification awarded}{Radio-amateur class A license}
\ecvitem{Principal subjects/occupational skills covered}{Course for radio-amateur
\begin{itemize}
    \item Excellent knowledge of radio technology basics
    \item Excellent knowledge of radio devices and its usage
    \item Basics of phisics and chemical (magnetism,  elettromagnetism)
\end{itemize}}
\ecvitem{Name and type of organisation providing education and training}{C.I.S.A.R. Foligno's section}
\ecvitem[10pt]{Level in national or international classification}{SUITABLE, International Callsign \textbf{IZ0OVB}}

\ecvitem{Dates}{\textbf{From 19 March 2007 to 23 March 2007}}
\ecvitem{Title of qualification awarded}{Certificate of attendance to Spanish course}
\ecvitem{Principal subjects/occupational skills covered}{\vspace{-2mm}
\begin{itemize}
    \item Basic grammar of Spanish
    \item The Spanish general culture
\end{itemize}}
\ecvitem{Name and type of organisation providing education and training}{Inhispania Intlance S.L / CIF:B83744847 , Montera 10-12, 1-1. 28013, Madrid (Spain)}
\ecvitem[10pt]{Level in national or international classification}{Valutazioni\footnote{Evaluation in accordance with the ``Common European Framework of Reference for Languages''}:
\begin{itemize}
    \item Speaking: A2
    \item Writing: A2
    \item Listening comprehension: A2
    \item Writing comprehension: A2
    \item Oral interaction: A2
\end{itemize}}

\ecvitem{Dates}{\textbf{03-10-25 March 2007}}
\ecvitem{Title of qualification awarded}{Certificate of attendance to Microsoft technology course}
\ecvitem{Principal subjects/occupational skills covered}{Topics covered in the course were:
\begin{itemize}
    \item .NET Framework Architeture (2.0)
    \item ADO.NET
    \item ASP.NET (web forms, Page, controls, security)
    \item C\#
    \item Web Service
    \item Ajax.net
    \item Microsoft Visual Studio 2005
\end{itemize}}
\ecvitem{Name and type of organisation providing education and training}{O.S.MO.S.IT, Via Strozzacapponi, 85, 06071 Castel del Piano (Pg)}
\ecvitem[10pt]{Level in national or international classification}{Basics of Microsoft programming}

\ecvitem{Dates}{\textbf{From September 2006 to February 2007}}
\ecvitem{Title of qualification awarded}{\textbf{UbuntuSemplice 6.06} book publishing with donation to Canonical Ltd.}
\ecvitem{Principal subjects/occupational skills covered}{\vspace{-2mm}
\begin{itemize}
    \item Contributor, author of many chapters and system engineer of \url{http://www.ubuntusemplice.org/}
    \item Debian virtual server administration
    \item Configuring and using the MediaWiki collaborative software for drawing and composition of the book
    \item Web server Apache configuration
    \item Mailing list configuration to manage the collaborative writing of the book 
\end{itemize}}
\ecvitem{Name and type of organisation providing education and training}{Ubuntusemplice Project - \url{http://www.ubuntusemplice.org/)}}
\ecvitem[10pt]{Level in national or international classification}{Excellent knowledge of Ubuntu}

\ecvitem{Dates}{\textbf{01-02-03 Dicember 2006}}
\ecvitem{Title of qualification awarded}{Certificate of attendance to the course: ISO certifications}
\ecvitem{Principal subjects/occupational skills covered}{Training on security and ISO certifications:
\begin{itemize}
    \item ISO 27001:2005
    \item Policy for Information Security
    \item Risks Analisys (RA)
    \item Analysis of controls of ISO 17799:2005
    \item Risk Transfer Process (RTP)
    \item Certification process
    \item Overview of existing certification audits
    \item Audit plan and checklist
    \item Audit report
    \item A look at the future certifications
\end{itemize}}
\ecvitem{Name and type of organisation providing education and training}{WEDOIT s.a.s. - Via Protomartiri Francescani, 26 - 06088 Assisi (PG), Italy}
\ecvitem[10pt]{Level in national or international classification}{Kwoledge of ISO certification process}

\ecvitem{Dates}{\textbf{From October 2002 to November 2006}}
\ecvitem{Title of qualification awarded}{\textbf{Laurea triennale (nuovo ordinamento) in Informatica}}
\ecvitem{Principal subjects/occupational skills covered}{Computer science Bachelor Degree, \textbf{address ``Network''}:
\begin{itemize}
    \item Mathematics (analytical and discrete)
    \item Programming (C, Java, Php, html, xml, xsl, dtd, Pascal, scripting bash and csh, VB.NET, VRML)
    \item Databases (Mysql, MS Access and related programming language)
    \item Networks (ATM, xDSL, Mpls, X.25, Frame Relay), types (wireless, wired) and interaction between them
    \item Knowledge of multimedia system
    \item Overview of parallel computing (mpi)
\end{itemize}}
\ecvitem{Name and type of organisation providing education and training}{University of Perugia, Computer Science department}
\ecvitem[10pt]{Level in national or international classification}{\textbf{102/110}}

\ecvitem{Dates}{\textbf{From September 1996 to June 2002}}
\ecvitem{Title of qualification awarded}{\textbf{Diploma in ragioniere programmatore (progetto Mercurio)}}
\ecvitem{Principal subjects/occupational skills covered}{Materie previste dal percorso di studio dell'Istituto Tecnico Commerciale, definito dal Ministero dell'Istruzione, ovvero:
\begin{itemize}
    \item Scienze della Materia
    \item Matematica e Laboratorio
    \item Scienze della Natura
    \item Trattamento Testi e Dati
    \item Seconda lingua straniera (Francese)
    \item Diritto ed Economia
    \item Economia Aziendale
    \item Economia Politica e Scienza delle Finanze
    \item Lingua e letteratura italiana
    \item Storia
    \item Informatica Gestionale
    \item Matematica applicata
    \item Prima lingua straniera (Inglese)
    \item Diritto
\end{itemize}}
\ecvitem{Name and type of organisation providing education and training}{Ministero della Pubblica Istruzione - I.T.C. ``Federico Cesi'', Terni}
\ecvitem[10pt]{Level in national or international classification}{\textbf{85/100}}

\ecvitem{Dates}{\textbf{From 2001 to 2002}}
\ecvitem{Title of qualification awarded}{Attestato di frequenza al Progetto Nazionale IFS (\textbf{Impresa Formativa Simulata})}
\ecvitem{Principal subjects/occupational skills covered}{Simulazione di un'impresa di smaltimento rifiuti, affiancati dall'impresa Iosa Carlo S.r.l. (\url{http://www.iosacarlo.com}).
Nell'ambito del progetto ho coordinato il lavoro di tutti gli studenti, realizzando l'organigramma dell'azienda simulata e programmando il sito dell'azienda.}
\ecvitem{Name and type of organisation providing education and training}{Ministero della Pubblica Istruzione - I.T.C. ``Federico Cesi'', Terni}
\ecvitem[10pt]{Level in national or international classification}{-}

\ecvitem{Dates}{\textbf{From 24 September 2001 to 14 October 2001}}
\ecvitem{Title of qualification awarded}{Attestato di frequenza al corso come tutor}
\ecvitem{Principal subjects/occupational skills covered}{Svoltasi l'attivit\'a di tutor/referente di un gruppo di altri 6 studenti/tutor, per le attivit\'a di POTENZIAMENTO DI ITALIANO delle prime classi, in ambito del progetto ``Accoglienza, Recupero, Potenziamento nelle Prime Classi''.
L'attivit\'a \'e consistita nell'affiancare i Docenti di Lettere al fine di offrire un valido supporto agli alunni delle Prime Classi nell'utilizzo del Computer, per poter eseguire attivit\'a di approfondimento con il mezzo multimediale, attraverso esercitazioni con un CD-Rom di Grammatica.}
\ecvitem{Name and type of organisation providing education and training}{Ministero della Pubblica Istruzione - I.T.C. ``Federico Cesi'', Terni}
\ecvitem[10pt]{Level in national or international classification}{Acquisite ottime capacit\'a relazionali, organizzative ed ottime competenze nell'insegnamento di materie tecniche}

\ecvitem{Dates}{\textbf{From 7 Dicember 2001 to 09 Dicember 2001}}
\ecvitem{Title of qualification awarded}{Attestato di frequenza del ``Pluto Meeting 2001''}
\ecvitem{Principal subjects/occupational skills covered}{Partecipazione all'organizzazione del ``Pluto Meeting 2001'', tenuto presso il suddetto istituto.}
\ecvitem{Name and type of organisation providing education and training}{Ministero della Pubblica Istruzione - I.T.C. ``Federico Cesi'', Terni}
\ecvitem[10pt]{Level in national or international classification}{-}

\ecvitem{Dates}{\textbf{From 26 March 2001 to 02 April 2001}}
\ecvitem{Title of qualification awarded}{Attestato di frequenza alla ``XI Settimana della cultura scientifica e tecnologica''}
\ecvitem{Principal subjects/occupational skills covered}{Partecipazione alla ``XI Settimana della cultura scientifica e tecnologica'', realizzando sia la locandina che il programma provinciale della settimana delle scienze in Adobe Photoshop 5.5 ed in Corel Draw 8.0, impegnandosi sia in orario curriculare che extra-curriculare pomeridiano con grande devozione, senso della responsabilit\'a, fungendo anche da punto di riferimento per tutti gli studenti del biennio che hanno partecipato al progetto.}
\ecvitem{Name and type of organisation providing education and training}{Ministero della Pubblica Istruzione - I.T.C. ``Federico Cesi'', Terni}
\ecvitem[10pt]{Level in national or international classification}{-}

\ecvitem{Dates}{\textbf{Year 2001}}
\ecvitem{Title of qualification awarded}{Attestato di frequenza come tutor al corso di alfabetizzazione di computer per over 65}
\ecvitem{Principal subjects/occupational skills covered}{Corso di alfabetizzazione di computer di base in funzione di tutor di 40 ore complessive ad un gruppo di 30 persone, con et\'a superiore al 65 anni.
Ho svolto inoltre il ruolo di coordinatore del progetto stilando il programma e coordinando i miei colleghi.}
\ecvitem{Name and type of organisation providing education and training}{Ministero della Pubblica Istruzione - I.T.C. ``Federico Cesi'', Terni}
\ecvitem[10pt]{Level in national or international classification}{-}

\ecvitem{Dates}{\textbf{Year 2001}}
\ecvitem{Title of qualification awarded}{Attestato di frequenza al corso di informatica}
\ecvitem{Principal subjects/occupational skills covered}{Corso di informatica di base in funzione di tutor (progetto 20 Studenti) di 30 ore complessive su applicazioni office (Word, Excel) ed Internet}
\ecvitem{Name and type of organisation providing education and training}{Ministero della Pubblica Istruzione - I.T.C. ``Federico Cesi'', Terni}
\ecvitem[10pt]{Level in national or international classification}{-}

\ecvitem{Dates}{\textbf{16-17-18 November 2000}}
\ecvitem{Title of qualification awarded}{Attestato di frequenza all'Exposcuola 2000}
\ecvitem{Principal subjects/occupational skills covered}{Exposcuola - Salone del confronto tra le proposte formative dell'Europa e del Mediterraneo - Paestum hotel Arison}
\ecvitem{Name and type of organisation providing education and training}{Ministero della Pubblica Istruzione - I.T.C. ``Federico Cesi'', Terni}
\ecvitem[10pt]{Level in national or international classification}{-}

\ecvitem{Dates}{\textbf{From 1997 to 1998}}
\ecvitem{Title of qualification awarded}{Attestato di frequenza}
\ecvitem{Principal subjects/occupational skills covered}{Corso di base sulla multimedialit\'a (progetto 20 studenti) per un totale di 25 ore.}
\ecvitem{Name and type of organisation providing education and training}{{Ministero della Pubblica Istruzione - I.T.C. ``Federico Cesi'', Terni}}
\ecvitem[10pt]{Level in national or international classification}{-}

\ecvsection{Personal skills and competences}

\ecvmothertongue[5pt]{Italian}
\ecvitem{\large Other language(s)}{\textbf{English, Spanish}}
\ecvlanguageheader{(*)}
\ecvlanguage{English}{\ecvBOne}{\ecvBOne}{\ecvATwo}{\ecvATwo}{\ecvBOne}
\ecvlastlanguage{Spanish}{\ecvBOne}{\ecvBOne}{\ecvATwo}{\ecvATwo}{\ecvBOne}
\ecvlanguagefooter[10pt]{(*)}

\ecvitem[10pt]{Social skills and competences}{Ottime capacit\'a di relazionarsi con colleghi e collaboratori. Socievole, simpatico e con buone doti comunicative. Propenso al lavoro in team. Il mio sito \'e fonte di contatti e scambi sociali continui con altre persone tecniche e meno techiche.)}
\ecvitem[10pt]{Organisational skills and competences}{Capacit\'a di lavorare in maniera efficiente in diverse situazioni.}
\ecvitem[10pt]{Computer skills and competences}{Vista la mia passione per l'informatica in generale ho sviluppato nell'arco degli anni una serie di competenze che variano in molti settori della stessa.

Fin dal mio primo computer ho avuto una certa passione per il mondo open source e tutto quello che lo riguarda: infatti ho amministrato macchine \textbf{Linux} con varie distribuzioni, come RedHat 7.3, Slackware 7.1 fino ad arrivare a macchine Debian (dalla versione 3.0 a quelle attuali).

Tramite questa esperienza ho maturato una certa abilit\'a e conoscenza nell' aministrazione di Linux: scripting bash, configurazione e compilazione del kernel, servizi di rete, patch per il kernel, linguaggio C. Oltre a Linux uso regolarmente \textbf{OSX}, per l'utilizzo quotidiano. Visto il continuo utilizzo e la mia passione per l'informatica ho approfondito lo studio di OSX, basato su sitemi UNIX.

\textbf{Inoltre ho una certa passione per quanto riguarda la programmazione}: conosco molti linguaggi in ambiti molto diversi tra di loro come Python, C, PHP, java, LSL (Linden Scripting Language). L'LSL l'ho studiato durante la mia attivit\'a su \textbf{Second Life}: infatti ho collaborato su molti progetti italiani presenti nel metaverso come Assisi \url{http://www.secundavita.it}, Milano e Marostica del progetto Italia Vera.
Ho una buona conoscenza di applicazioni grafiche (Gimp, Photoshop)e di strumenti per l'ufficio come Openoffice,org ed iWork (per OSX)}
\ecvitem[10pt]{Artistic skills and competences}{\vspace{-2mm}
\begin{itemize}
    \item Apprendimento della lingua spagnola da autodidatta
    \item Foto amatoriale
    \item Musica (livello hobbystico)
\end{itemize}}

\ecvitem[10pt]{Other skills and competences}{\vspace{-2mm}
\begin{itemize}
    \item Parkour
    \item Musica
    \item Propenso all'apprendimento ed allo studio
    \item Attrazione per le materie scientifiche in generale
\end{itemize}}

\ecvitem{Licence(s)}{\vspace{-2mm}
\begin{itemize}
    \item Patente di Guida B.
    \item Patente di Operatore di stazione di radioamatore di classe A (nr. 020122/AN), nominativo Internazionale \textbf{IZ0OVB}
\end{itemize}}

\ecvsection{Additional information}

\ecvitem[10pt]{}{\vspace{-10mm}
\begin{itemize}
    \item in regola con gli obblighi di leva (rinvio per studio)
    \item Linux Registered User \#399008
    \item socio ordinario del LUG di Perugia
    \item socio ordinario dell'AVIS, sia come consigliere comunale di Acquasparta sia come Gruppo Avis Giovani
    \item stato civile: celibe
\end{itemize}}

\ecvsection{Attachment}
\ecvitem{}{No attachments}

\end{europecv}
\end{document} 