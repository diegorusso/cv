% possible options include font size ('10pt', '11pt' and '12pt'),
% paper size ('a4paper', 'letterpaper', 'a5paper', 'legalpaper',
% 'executivepaper' and 'landscape') and font family ('sans' and 'roman')
\documentclass[10pt,a4paper,sans]{moderncv}

% moderncv themes
% style options are 'casual' (default), 'classic', 'banking', 'oldstyle' and
% 'fancy'
\moderncvstyle{classic}
% color options 'black', 'blue' (default), 'burgundy', 'green', 'grey',
% 'orange', 'purple' and 'red'
\moderncvcolor{burgundy}
% to set the default font; use '\sfdefault' for the default sans serif font,
% '\rmdefault' for the default roman one, or any tex font name
% \renewcommand{\familydefault}{\sfdefault}
% uncomment to suppress automatic page numbering for CVs longer than one page
% \nopagenumbers{}

% character encoding
% if you are not using xelatex or lualatex, replace by the encoding you are
% using
\usepackage[utf8]{inputenc}
% if you need to use CJK to typeset your resume in Chinese, Japanese or Korean
% \usepackage{CJKutf8} 

\usepackage{footmisc} % enabling footnotes.

% adjust the page margins
\usepackage[scale=0.9]{geometry}
% if you want to change the width of the column with the dates
\setlength{\hintscolumnwidth}{2.65cm}
% for the 'classic' style, if you want to force the width allocated to your
% name and avoid line breaks. be careful though, the length is normally
% calculated to avoid any overlap with your personal info; use this at your own
% typographical risks...
% \setlength{\makecvtitlenamewidth}{10cm}

% required when changes are made to page layout lengths
\AtBeginDocument{\recomputelengths}

% personal data
\name{Diego}{Russo}
% Below fields are optional, remove the line if not wanted
\title{Software Engineer}
% {postcode city}{country}; the "postcode city" and "country" arguments can be
% omitted or provided empty
\address{Cambridge, UK}
% the optional "type" of the phone can be "mobile" (default), "fixed" or "fax"
% \phone[fixed]{+2~(345)~678~901}
% \phone[fax]{+3~(456)~789~012}
\phone[mobile]{+44 (0) 7428 251191}
\email{me@diegor.it}
\homepage{www.diegor.uk}
\social[linkedin]{diegor}
\social[twitter]{diegor}
\social[github]{diegorusso}
\extrainfo{last update: April 2021}
% '64pt' is the height the picture must be resized to, 0.4pt is the thickness
% of the frame around it (put it to 0pt for no frame) and 'picture' is the name
% of the picture file
\photo[80pt][0.4pt]{images/diegor}
\quote{``Now is better than never'' --- The Zen of Python}

% bibliography adjustments (only useful if you make citations in your resume,
% or print a list of publications using BibTeX)
% to show numerical labels in the bibliography (default is to show no labels)
%\makeatletter
%\renewcommand*
%    {\bibliographyitemlabel}
%    {\@biblabel{\arabic{enumiv}}}
%\makeatother
\renewcommand*{\bibliographyitemlabel}{[\arabic{enumiv}]}
% to redefine the bibliography heading string ("Publications")
% \renewcommand{\refname}{Articles}

% bibliography with mutiple entries
% \usepackage{multibib}
% \newcites{book,misc}{{Books},{Others}}
% -----------------------------------------------------------------------------
%            content
% -----------------------------------------------------------------------------
\begin{document}
% -----       resume       ----------------------------------------------------
\makecvtitle\section{Desired employment}

\cvline{}{\large\textbf{I'm always looking for a challenging position where I
can express and use my passion for programming and technology. I mostly develop
in Python in *NIX environment on a daily basis. As I'm a dynamic person,
professional and personal growth are very important to me.}}

\section{Experience}
\subsection{Vocational}
% arguments 3 to 6 can be left empty
% \cventry{year--year}{Degree}{Institution}{City}{\textit{Grade}}{Description}
\cventry{\textbf{Since 2021--04}}
    {Principal Software Engineer in ML Group}
    {Arm Ltd, \url{http://www.arm.com}}
    {Cambridge, UK}{}
    {\textbf{I'm leading the AI evaluation toolkit project}: due its
    complexity, the project requires efforts from team across ML and Arm.
    I'm acting as central point amongst these teams, coordinating both
    technical tasks and communications.
    I also did start a \textbf{Python Guild} which counts almost 500 members
    across the company and I've been an active member of the \textbf{ML
    Software Quality Forum.}}

\cventry{2020--01/2021--03}
    {Staff Software Engineer in ML Group}
    {Arm Ltd, \url{http://www.arm.com}}
    {Cambridge, UK}{}
    {Working in the ML tooling team, I used my Python expertise across a range
    of projects. I moved then to the \textbf{AI Evaluation Toolkit} for
    enabling customers to evaluate Arm IP for ML workloads.
    In the meantime I finished my master's exams and worked on the master
    thesis.}

\cventry{2017--09/2019--12}
    {Staff Software Engineer in ISG}
    {Arm Ltd, \url{http://www.arm.com}}
    {Cambridge, UK}{}
    {Working in the \textbf{Mbed Linux OS team}, \textbf{I lead the build, test
    and validation automation infrastructure of our product: from build to data
    representation} while being part of the core engineer.
    My day to day job required constant communication with other teams overseas
    and with external partners.
    Software stack used: \textbf{Jenkins/pipelines, LAVA, Artifactory, Docker,
    Python, Bash, Linux, GitHub.}}

\cventry{2016--06/2017--08}
    {Staff Software Engineer in DSG}
    {Arm Ltd, \url{http://www.arm.com}}
    {Cambridge, UK}{}
    {\textbf{Leading an infrastructure team (8)}, I looked after both the Arm
    Compiler and OSS Toolchains automation infrastructures (GNU-A, GNU-RM and
    LLVM OSS). Besides the technical lead, my role was to make sure all
    infrastructures respect design principles whilst sharing knowledge,
    technologies and processes across different teams. I kept a monthly forum
    with stakeholders to be sure to catch their requirements and I represented
    DSG infrastructures across the business.}

\cventry{2013--11/2016--05}
    {Staff Software Engineer in DSG}
    {Arm Ltd, \url{http://www.arm.com}}
    {Cambridge, UK}{}
    {Working in the GNU team, our goal is the delivery of the GNU toolchain
    optimized for Arm processors. \textbf{I was responsible of the
    infrastructure for automatic builds, tests and benchmarks.} Most of the
    infrastructure was \textbf{Python} based and it automated all the phases of
    the process, and interacting with board farm.
    I developed an application for triaging GCC test results: it stores them
    into a database in order to analyse those data and to find error root
    causes, trends, common patterns and automating the tracking of issues using
    Jira.
    Software stack used: \textbf{Python, flask, bootstrap, AngularJS,
    MongoDB, Linux, svn/git.}}

\cventry{2011--10/2013--10}
    {Senior Software Engineer in Engineering IT}
    {Arm Ltd, \url{http://www.arm.com}}
    {Cambridge, UK}{}
    {I worked on a variety of internal projects. I developed a reliable and
    fault tolerant application that interacts with the LFS cluster and a AMQ
    server.
    I also developed a Jira plug-in to interact with an internal software in
    order to synchronise external tickets with internal ones.
    I looked after, improved and fixed many internal IT systems.
    I ran the \textit{IT Early Career Scheme} for hiring graduates and interns.
    Software stack used: \textbf{Python, MongoDB, RabbitMQ, Java, Jira, LSF,
    Perl, C, tcsh, bash, Linux, svn/git.}}

\cventry{2008--04/2011--02}
    {Software/System Engineer in R\&D}
    {Forinicom Srl, \url{http://www.com-com.it}}
    {Bastia Umbra, IT}{}
    {Worked in the R\&D team to create an innovative product in the wireless
    communications market. I initially worked on \textbf{embedded systems},
    customizing the OS and the software to manage authentication. I focused
    then on the software for deploying these devices and on large-scale
    production environment. Also, we developed a complete solution for managing
    \textbf{hotspots}: I worked on a captive-portal to manage authentication,
    sessions log, signs up, signals management from remote devices, integration
    with our management software, payment via credit card and authentication
    via SMS\@. Finally I did create a software for network monitoring.
    Software stack used: \textbf{Python/Django, PostgreSQL database, Debian,
    XEN, PyQT, git.}}

\cventry{2010--11/2011--01}
    {Python Engineer}
    {Exion Sagl, Manno, Switzerland, \url{http://www.exion.ch/}}
    {Remote, Assisi, IT}{}
    {Setting up an \textbf{Adult WebTV} entirely developed in
    \textbf{Python/Django with PostgreSQL} database on Linux/Apache platform
    and Red5 as streaming server.}

\cventry{2010--10/2011--01}
    {Python Engineer}
    {Sauce Labs Inc, San Francisco, USA, \url{http://saucelabs.com}}
    {Remote, Assisi, IT}{}
    {Implementing new features, bug fixing, structural changes to the site of
    Sauce Labs. Distance work coordination. Software stack used:
    \textbf{Python, Pylons, GitHub.}}

\cventry{2006--12/2008--08 2009--09/2011--09}
    {Python/Django Engineer}
    {Consorzio Miles --- Servizi Integrati, Rome}
    {Assisi, IT}{}
    {Working in a team, I developed a management application for the
    municipality of Bettona, for the computerization of services, the
    management of personal data, building practices, urban planning,
    calculation of ICI tax and updating of land registry data. Also I created
    an advanced web interface for sending proposed practices, on-line services
    conference, integration process, exploration of cadastral map in
    \textbf{DXF} and production of customized automated printing.
    \textbf{The project had a gap year when I started working for Forinicom
    Srl.} Whenever there are overlaps, the jobs have been on a part-time
    basis.
    Software stack used: \textbf{Django, Python, PostgreSQL, Linux, Apache,
    svn/git, trac.}}

\subsection{Miscellaneous}
\cventry{2016--01/2016--06}
    {iPhone (iOS9) and Swift 2.2}
    {University of Perugia, \url{http://dmi.unipg.it}}
    {Udemy --- Cambridge, UK}{}
    {After self learning Swift 2.2, I attended an online course to create iOS
    apps. I used those skills to create DipMatBeacon
    (\url{https://github.com/diegorusso/DipMatBeacon}), an app used to check
    the booking state of Maths department rooms using iBeacon technology.}

\cventry{2015--06/2015--12}
    {mm\_mpi: MPI Matrix multiplication}
    {University of Perugia, \url{http://dmi.unipg.it}}
    {Cambridge, UK}{}
    {Developed an application in C using MPI and Cannon's algorithm. The code
    generates different versions of the application: blocking and non-blocking
    MPI calls, CBLAS, and different OpenMP optimizations. Codebase:
    \url{https://github.com/diegorusso/mm\_mpi}}

\cventry{2013--06/2015--06}
    {Staff Software Engineer}
    {Opentaste Ltd \url{http://www.opentaste.eu}}
    {Remote, Cambridge, UK}{}
    {Working in my spare time, I was part of a distributed team of 10 people
    across the globe (San Francisco, Italy, Australia). My main tasks were
    technical advisor and code reviewer. We did organise regular hangouts
    to discuss about plans, issues and tasks.
    Software stack used: \textbf{Python, Flask, MongoDB, GitHub, Google Docs.}}

\cventry{2011--06}
    {Computer course teacher}
    {Centro Studi Foligno, \url{http://www.cstudifoligno.it}}
    {Foligno, IT}{}
    {Taught a class of 10 people the existence of the open source world,
    installing open source software on Windows and then proceed to install
    Ubuntu on their laptop.}

\cventry{2011--05/2011--06}
    {Objective-C Engineer}
    {Forinicom Srl, Bastia Umbra, \url{http://www.com-com.it}}
    {Assisi, IT}{}
    {Developed an iPhone app that permits you to auto-login into ComCom
    Hotspot. This application was used at Europython 2011 in Florence.}

\cventry{2011--01/2011--06}
    {iPhone (iOS4) and Objective-C}
    {University of Perugia, \url{http://dmi.unipg.it}}
    {Assisi, IT}{}
    {Attended online Stanford University classes, I learnt Objective-C,
    developing simple applications. As final project I customized a VOIP
    application for iPhone based on Linphone (\url{http://www.linphone.org}).}

\cventry{2009--08/2010--03}
    {Paper publication\cite{aes}}
    {University of Perugia, \url{http://dmi.unipg.it}}
    {Perugia, IT}{}
    {Preparation and publication of ``The AES implementation based on OpenCL
    for multi/many core architecture'' paper for the yearly conference ICCSA
    2010 (\url{www.iccsa.org}) at Sangyo University, Fukuoka in Japan. The
    paper discussed the implementation of an AES algorithm that runs on
    NVIDIA/ATI graphics card.}

\cventry{2005--11/2006--05}
    {SEO Search Engine Optimization}
    {WEDOIT sas}
    {Assisi, IT}{}
    {Working in a team I acquired knowledge of SEO and its behaviour. The
    internship included SEO optimisation of various websites, using
    \emph{pagerank} and \emph{link popularity} methods. Also I worked as a
    system engineer of Debian-based virtualised server and I developed a SEO
    oriented application in Python and PHP.}

\cventry{2002--02}
    {Internship combined with IFS project, Impresa Formativa Simulata
        (Enterprise Training Simulation)}
    {IOSA CARLO Srl}
    {Terni, IT}{}
    {Administration of enterprise network and coordination of the students
    work, building the simulated organization chart and developing the
    website.}

\section{Trainings}
\cventry{\textbf{Since 2011}}
    {Europython Conference}
    {Volunteer, Attendee}
    {Florence, Berlin, Bilbao, Rimini, Edinburgh, online}
    {}
    {I volunteered in the 2011 edition and attended since then.
    Europython is my annual appointment with the Python world as it is
    a source of ideas that keeps me updated with the latest trends around
    Python.}

\cventry{2015--10}
    {MongoDB Administration and Developers training}
    {MongoDB}
    {Arm Ltd --- Cambridge, UK}{}
    {3 days training covering MongoDB essentials (query language, indexes,
    aggregation framework, replica set, sharding\dots)}

\cventry{2014--01/2014--03}
    {Cryptography 1}
    {Coursera}{Online --- Cambridge, UK}
    {Final score: \textbf{PASSED}}
    {Topics covered in the course: semantic security, block ciphers and
    pseudorandom functions, DES/AES block ciphers, message integrity,
    collision resistant hashing, authenticated encryption, Diffie-Hellman, RSA,
    and Merkle puzzles, public key encryption, the ElGamal system
    (\url{http://bit.ly/coursera-crypto1})}

\cventry{2013--09}
    {ITIL Foundation}
    {ILX Group}{
    Arm Ltd, Cambridge, UK}
    {Foundation Score: \textbf{31 out of 40, PASSED}}
    {The skills achieved are all included in
    (\url{http://bit.ly/itil-foundation})}

\cventry{2007--02/2007--07}
    {Radio-amateur \textbf{class A license}}
    {C.I.S.A.R. Foligno's section}
    {Assisi, IT}
    {\textbf{PASSED}}
    {During the course for radio-amateur I acquired excellent knowledge of
    radio technology basics, radio devices and its usage and basics of
    magnetism, electromagnetism. Operator license for amateur radio station
    class A (nr. 020122/AN), International callsign \textbf{IZ0OVB}}

\section{Education}
\cventry{2010--10/2021--04}
    {Master Degree in Computer Science, ``Security''}
    {University of Perugia, \url{http://dmi.unipg.it}}
    {Perugia, IT/Cambridge, UK}
    {\textbf{IT\@: 110/110 cum laude --- UK\@: First class honours}}
    {While working full-time, I did study during my spare time. Topics covered:
    AI, parallel computing, simulation, advanced programming and operating
    systems, theoretical computer science, computer security, advanced
    databases and data mining, law applied to computer science and
    communications, cryptography, mobile programming, operations research.}

\cventry{2002--10/2006--11}
    {Bachelor Degree in Computer Science, ``Networking''}
    {University of Perugia, \url{http://dmi.unipg.it}}
    {Perugia, IT}
    {\textbf{IT\@: 102/110 --- UK\@: 2:1}}
    {Topics covered: Mathematics (analytical and discrete), Programming (C,
    Java, Php, html, xml, xsl, dtd, Pascal, bash/csh, VB.NET, VRML), Databases
    (Mysql, MS Access), Networks (ATM, xDSL, Mpls, X.25, Frame Relay) types
    (wireless, wired) and interaction between them, Knowledge of multimedia
    system, overview of parallel computing (mpi)}

\cventry{1996--09/2002--06}
    {Accountant programmer Diploma (Mercurio project)}
    {Ministry of Public Education --- I.T.C. (Commercial technical institute)
        ``Federico Cesi''}
    {Terni, IT}
    {\textbf{IT\@: 85/100 --- UK\@: A}}
    {Topics covered by the course (Commercial technical institute) of study as
    defined by the Ministry of Public Education: Chemistry / Physics,
    Mathematics and Computer Laboratory, Natural Science, Word processing and
    data, Second foreign language (French), Law and Economics, Business,
    Economics and Financial Science, Italian Language and Literature, History,
    Computer Management, Applied Mathematics, First foreign language (English),
    Law.}

\section{Master Thesis}
\cvline{title}
    {\textbf{Per-Layer Pruning With Heuristic} --- \emph{Pruning layers of a
        neural network with a heuristic-based approach}
        (2020--09/2021--04)}
\cvline{supervisors}{Alfredo Milani, Anton Kachatkou}
\cvline{description}
    {The aim of this thesis is to investigate and validate a specific pruning
    technique: \textbf{layers of a network model are pruned based on a
    heuristic whilst respecting the target sparsity of the network model.}
    The code is implemented in TFMOT and experimented on MobileNet v1 with
    CIFAR-10 and ImageNet 2012 datasets.
    The heuristic distribution of weights behave more robustly compared to the
    uniform distribution especially with higher sparsity levels.}

\section{Bachelor Thesis}
\cvline{title}
    {\textbf{Wireless Broadband Network} --- \emph{Weconnect project}
        (2006--07/2006--12)}
\cvline{supervisors}{Simone Brunozzi, Sergio Tasso}
\cvline{description}
    {The thesis was to develop a Wi-Fi network in order to cover
    \textbf{digital-divide} areas. Thanks to this project, I acquired a wide
    knowledge about Wi-Fi networks, legislation that governs the operation,
    RouterOS (\url{www.mikrotik.com}), AAA protocol and Radius server. I
    managed a server for the provision of various network services: mail
    (Postfix), web server (Apache), DNS (pdns), firewall (iptables), database
    (PostgreSQL), hotspot (Chillispot), Debian OS, Voyage OS.}

\section{Languages}
\cvitemwithcomment{Italian}
    {\textbf{Mother tongue}}{Italian citizenship}
\cvitemwithcomment{English}
    {\textbf{C1 level}}{British citizenship}
\cvitemwithcomment{Spanish}
    {\textbf{C1 level}}{}

\section{Interests}
\cvline{Technology}
    {\small Attracted by all things which have a processor}
\cvline{Music}
    {\small Hobby level. I played piano and guitar and I love to listen to any
    genre, from salsa to metal}
\cvline{Sports}
    {\small Cuban Salsa dancing. Past: Taichi, Squash, Capoeira, Kungfu,
    Swimming}
\cvline{Puzzles}
    {\small I love any kind of puzzle and I'm very passionate about Rubik's
    cubes: sizes solved from $2\times2\times2$ to $9\times9\times9$}

% Publications from a BibTeX file without multibib
\nocite{*}
\bibliographystyle{plainurl}
\bibliography{diego_russo_publications}

% Publications from a BibTeX file using the multibib package
% \section{Publications}
% \nocitebook{book1,book2}
% \bibliographystylebook{plain}
% 'publications' is the name of a BibTeX file
% \bibliographybook{publications}
% \nocitemisc{misc1,misc2,misc3}
% \bibliographystylemisc{plain}
% 'publications' is the name of a BibTeX file
% \bibliographymisc{publications}

\section{About this CV}
\cvdoubleitem{Language}{Proudly written in \LaTeX\ and Vim}{Source}{\url{https://github.com/diegorusso/cv}}
%\input{revision}
%\cvline{Commit}{\Revision}

\clearpage
% -----     letter    ---------------------------------------------------------
% recipient data
% \recipient{Company Recruitment team}{Company, Inc. 123 somestreet, some city}
% \date{January 01, 1984}
% \opening{Dear Sir or Madam,}
% \closing{Yours faithfully,}
% use an optional argument to use a string other than "Enclosure", or 
% redefine \enclname
% \enclosure[Attached]{curriculum vit\ae{}}
% \makelettertitle

% \makeletterclosing
% if you are typesetting your resume in Chinese using CJK; the \clearpage is
% required for fancyhdr to work correctly with CJK, though it kills the page
% numbering by making \lastpage undefined
% \clearpage\end{CJK*}
\end{document}
