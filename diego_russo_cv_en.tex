% Ensure hyperlink colors are enabled when moderncv loads hyperref.
\PassOptionsToPackage{colorlinks=true,linkcolor=red,urlcolor=red,citecolor=red}{hyperref}
% possible options include font size ('10pt', '11pt' and '12pt'),
% paper size ('a4paper', 'letterpaper', 'a5paper', 'legalpaper',
% 'executivepaper' and 'landscape') and font family ('sans' and 'roman')
\documentclass[10pt,a4paper,sans]{moderncv}

% moderncv themes
% color options 'black', 'blue' (default), 'burgundy', 'green', 'grey',
% 'orange', 'purple' and 'red'
\moderncvcolor{red}
% style options are 'casual' (default), 'classic', 'banking', 'oldstyle' and
% 'fancy'
\moderncvstyle[nosymbols]{banking}
% Use Font Awesome symbols without requiring XeTeX/LuaTeX.
\moderncvicons{awesome}
% to set the default font; use '\sfdefault' for the default sans serif font,
% '\rmdefault' for the default roman one, or any tex font name
% \renewcommand{\familydefault}{\sfdefault}
% uncomment to suppress automatic page numbering for CVs longer than one page
% \nopagenumbers{}

% character encoding
% if you are not using xelatex or lualatex, replace by the encoding you are
% using
\usepackage[utf8]{inputenc}
% Avoid cmss font shape warnings by mapping missing shapes to existing ones.
\DeclareFontShape{T1}{cmss}{b}{n}{<->ssub * cmss/bx/n}{}
\DeclareFontShape{T1}{cmss}{b}{it}{<->ssub * cmss/bx/it}{}
\DeclareFontShape{T1}{cmss}{sb}{n}{<->ssub * cmss/m/n}{}
% if you need to use CJK to typeset your resume in Chinese, Japanese or Korean
% \usepackage{CJKutf8}

% adjust the page margins
\usepackage[scale=0.92]{geometry}
% if you want to change the width of the column with the dates
\setlength{\hintscolumnwidth}{2.65cm}
% for the 'classic' style, if you want to force the width allocated to your
% name and avoid line breaks. be careful though, the length is normally
% calculated to avoid any overlap with your personal info; use this at your own
% typographical risks...
% \setlength{\makecvtitlenamewidth}{10cm}

% required when changes are made to page layout lengths
\AtBeginDocument{\recomputelengths}
% Match moderncv dotted subsection rule, but avoid underfull hbox warnings.
\renewcommand*{\subsectionrule}{\leavevmode{\color{bodyrulecolor}\xleaders\hbox to 0.35em{\hss\scriptsize.\hss}\hfill}}

% personal data
\name{}{Diego Russo}
% Below fields are optional, remove the line if not wanted
\title{Principal Software Engineer | CPython Core Developer}
% {postcode city}{country}; the "postcode city" and "country" arguments can be
% omitted or provided empty
\email{me@diegor.it}
\phone[mobile]{+44 (0) 7428 251191}
\address{Cambridge, UK}
% the optional "type" of the phone can be "mobile" (default), "fixed" or "fax"
% \phone[fixed]{+2~(345)~678~901}
% \phone[fax]{+3~(456)~789~012}
\homepage{www.diegor.it}
\social[linkedin]{diegor}
\social[twitter]{diegor}
\social[github]{diegorusso}
\extrainfo{Updated on \today}
% '64pt' is the height the picture must be resized to, 0.4pt is the thickness
% of the frame around it (put it to 0pt for no frame) and 'picture' is the name
% of the picture file
%\photo[80pt][0.4pt]{images/diegor}
%\quote{``Now is better than never'' --- The Zen of Python}

% bibliography adjustments (only useful if you make citations in your resume,
% or print a list of publications using BibTeX)
% to show numerical labels in the bibliography (default is to show no labels)
%\makeatletter
%\renewcommand*
%    {\bibliographyitemlabel}
%    {\@biblabel{\arabic{enumiv}}}
%\makeatother
\renewcommand*{\bibliographyitemlabel}{[\arabic{enumiv}]}
% to redefine the bibliography heading string ("Publications")
% \renewcommand{\refname}{Articles}

% bibliography with mutiple entries
% \usepackage{multibib}
% \newcites{book,misc}{{Books},{Others}}

%Colour links in the CV
% hyperlink colors configured via \PassOptionsToPackage above

% -----------------------------------------------------------------------------
%            content
% -----------------------------------------------------------------------------
\begin{document}
% -----       resume       ----------------------------------------------------
\maketitle
CPython Core Developer and Principal Software Engineer in Arm's Runtimes team,
based in Cambridge, UK. 20+ years in software engineering, using Python since
2006 and contributing to CPython since 2023, with a focus on interpreter
performance, JIT-related work, CI infrastructure, and ensuring CPython and its
ecosystem run reliably and efficiently on Arm platforms. Work sits at the
intersection of runtime, performance engineering, and large-scale open-source
collaboration. EuroPython organiser and lead of the Arm Python Guild, an
internal community of more than 1,400 Python developers across the company.

\section{Career Experience}
% arguments 3 to 6 can be left empty
% \cventry{year--year}{Degree}{Institution}{City}{\textit{Grade}}{Description}

%\cventry{}
%    {Position}
%    {Group, Company, Location, Country}
%    {\textbf{Period}}
%    {}
%    {\textbf{Technical Scope}: Python\newline
%    Stuff I did
%    \begin{itemize}
%        \item Stuff I did to highlight
%    \end{itemize}}

\cventry{}
    {Principal Software Engineer}
		{CE-SW Runtimes, Arm Ltd., Cambridge, UK}
    {\textbf{2023--present}}
    {}
    {\textbf{Technical Scope}: Tech Lead, CPython.\newline
		Started with a six-month secondment in 2023 to assess the CPython
		ecosystem on Arm platforms, then joined the Runtimes team to continue
		enablement work with upstream collaboration and ecosystem reliability
		for developers. Worked on CPython JIT-related investigations and
		performance work on Arm platforms.
    \begin{itemize}
			\item Organised the CPython Core Dev Sprint 2025 at Arm in Cambridge
				(\href{https://developer.arm.com/community/arm-community-blogs/b/tools-software-ides-blog/posts/cpython-core-dev-sprint-2025-at-arm-cambridge-the-biggest-one-yet}{blog post}).
    \end{itemize}}

\cventry{}
    {Principal Software Engineer/Staff Software Engineer}
    {ML Group, Arm Ltd., Cambridge, UK}
    {\textbf{2020--2023}}
    {}
    {\textbf{Technical Scope}: Tech Lead, Python, Machine Learning, IP
    Evaluation, Inference Advisor, Software Quality.\newline
    Led the open-source \href{https://pypi.org/project/mlia/}{Arm ML Inference
    Advisor} project and coordinated technical delivery across ML and Arm
    teams. Owned design, implementation, and maintenance of ML tooling and
    partner integrations. Previous project: IP Selection Sandbox for ML\@.
    \begin{itemize}
        \item Helped AI/ML developers tailor and optimize models for Arm
            hardware: \url{https://pypi.org/project/mlia/}
        \item Built the IP Selection Sandbox for ML, enabling clients to
            evaluate the right IP for workloads.
    \end{itemize}}

\cventry{}
    {Staff Software Engineer}
    {ISG, Arm Ltd., Cambridge, UK}
    {\textbf{2017--2019}}
    {}
    {\textbf{Technical Scope}: Jenkins/Pipelines, LAVA, Artifactory, Docker,
    Python, Bash, Linux, GitHub.\newline
    Led build, test, data representation, and validation automation for
    product development. Partnered with global teams and external partners,
    produced system design documents, and improved CI reliability.
    \begin{itemize}
        \item Designed automated build and test infrastructure from scratch on
            real Cortex-A boards for custom Linux distribution.
    \end{itemize}}

\cventry{}
    {Staff Software Engineer}
    {DSG, Arm Ltd., Cambridge, UK}
    {\textbf{2013--2017}}
    {}
    {\textbf{Technical Scope}: Python, Flask, Bootstrap, AngularJS, MongoDB,
    Linux, SVN/Git\newline
    Supervised an eight-member team and managed automation infrastructure for
    Arm Compiler and OSS toolchains. Ensured compliance with design principles,
    shared knowledge across teams, and hosted a monthly stakeholder forum.
    Acted as business representative on DSG infrastructures.
    \begin{itemize}
        \item Collaborated with the GNU team to deliver optimized GNU
            toolchains for Arm processors.
        \item Drove development of infrastructure for automated builds, tests,
            and benchmarks.
        \item Created an application to triage GCC test results, store results
            in a database, and automate issue tracking using Jira.
    \end{itemize}}

\cventry{}
    {Senior Software Engineer}
    {Engineering IT, Arm Ltd., Cambridge, UK}
    {\textbf{2011--2013}}
    {}
    {\textbf{Technical Scope}: Python, MongoDB, RabbitMQ, Java, Jira, LSF,
    Perl, C, tcsh, bash, Linux, SVN/Git\newline
    Contributed to internal platforms, including a fault-tolerant application
    integrating with LSF cluster and AMQ server. Built a Jira plug-in to
    synchronize external and internal tickets, and maintained core IT systems.
    \begin{itemize}
        \item Planned and executed the IT Early Career Scheme for interns and
            graduates.
    \end{itemize}}

\cventry{}
    {Software/System Engineer}
    {R\&D, Forinicom Srl, Bastia Umbra, IT}
    {\textbf{2008--2011}}
    {}
    {\textbf{Technical Scope}: Python/Django, PostgreSQL, Debian, XEN, PyQT,
    Git\newline
    Delivered embedded systems and web solutions for authentication and
    hotspot management. Built a captive portal for sign-up, session logs,
    remote device signals, and payment integration, plus network monitoring
    software.
    \begin{itemize}
        \item Set up internet connection and enabled fast connectivity for
            rural areas, providing services to hundreds of clients.
        \item Designed and implemented hotspot system with captive portal and
            credit card payments for numerous tourists; used at EuroPython
            2011.
    \end{itemize}}

\cventry{}
    {Python/Django Engineer}
    {Consorzio Miles, Servizi Integrati, Rome/Assisi, IT}
    {\textbf{2006--2008}}
    {}
    {\textbf{Technical Scope}: Python/Django, PostgreSQL, Linux\newline
    Worked with a team to replace paper-based processes with a Django web
    system. Led migration from VB to Python/Django, enabling citizens to
    track and update municipal processes online in real time.}

\subsection{Additional Experience}
%\cventry{}
%    {Position}
%    {Company, Location, Country}
%    {\textbf{Period}}
%    {\textbf{Technical Scope}: Python}
%    {}

\cventry{Contract}
    {Objective-C Engineer}
    {Forinicom Srl., Bastia Umbra, IT}
    {\textbf{2011--05 / 2011--06}}
    {}
    {\textbf{Technical Scope}: Hotspot Captive Portal app, used at EuroPython
    2011 in Florence}


\cventry{Remote, Contract}
    {Python Engineer}
    {Exion Sagl, Manno, CH}
    {\textbf{2010--11/2011--01}}
    {}
    {\textbf{Technical Scope}: Python, Django, PostgreSQL}

\cventry{Remote, Contract}
    {Python Engineer}
    {Sauce Labs Inc., San Francisco, USA}
    {\textbf{2010--10/2011--01}}
    {}
    {\textbf{Technical Scope}: Python, Pylons, GitHub}

\cventry{Internship}
    {SEO Engineer}
    {Wedoit Sas., Assisi, IT}
    {\textbf{2005--11 / 2006--05}}
    {}
    {\textbf{Technical Scope}: SEO, Python, PHP}

\section{Talks \& Podcasts}
\cvitemwithcomment{2026}{Python SDQ January 2026 Meetup: Python por dentro: personas, procesos, código}
    {\href{https://github.com/diegorusso/diegorusso/blob/main/2026/python_sdq_python_por_dentro.pdf}{Slides} |
    \href{https://www.youtube.com/watch?v=PbLSttImQ3k}{YouTube}}
\cvitemwithcomment{2025}{EuroPython / PyCon UK: Exploring the CPython JIT}
    {\href{https://github.com/diegorusso/diegorusso/blob/main/2025/ep2025_exploring_the_cpython_jit.pdf}{Slides} |
    \href{https://youtu.be/5-AA7-fHYYM?si=3jILxr9B1Coeq3-F}{PyCon UK} |
    \href{https://youtu.be/5si4zkAngpA?si=cxYDkwUK7w-r1jJQ}{EuroPython}}
\cvitemwithcomment{2025}{Microsoft Build: Run PyTorch natively on Windows on Arm using GitHub runners}
    {\href{https://build.microsoft.com/en-US/sessions/ODFP974}{Session} |
    \href{https://www.youtube.com/watch?v=uhGg7wb6jV4}{Recording}}
\cvitemwithcomment{2025}{Intervista Pythonista: Runtime e Interpreter con un Core Developer! \#66}
    {\href{https://www.youtube.com/watch?v=Xx09D359mR4}{YouTube} |
    \href{https://creators.spotify.com/pod/profile/marco-santoni/episodes/Runtime-e-Interpreter-con-un-Core-Developer--66-e35gcjl}{Spotify} |
    \href{https://podcasts.apple.com/fr/podcast/intervista-pythonista/id1561566952}{Apple Podcasts}}
\cvitemwithcomment{2024}{Arm Innovation Coffee: GitHub Arm-hosted Runners with Larissa Fortuna and Diego Russo}
    {\href{https://www.youtube.com/watch?v=CMO4rL2msoQ}{YouTube}}
\cvitemwithcomment{2023}{EuroPython: Python on Arm Architecture}
    {\href{https://github.com/diegorusso/diegorusso/blob/main/2023/ep2023_python_on_arm.pdf}{Slides} |
    \href{https://www.youtube.com/watch?v=nYf7r0lkTIs}{Recording}}

\section{Education}

\cventry{}
    {University of Perugia, Perugia, IT}
    {Master of Science in Computer Science (Security) IT\@: 110/110 cum laude
        --- UK\@: First class honours}
    {\textbf{2021}}
    {}
    {Thesis: \href{https://github.com/diegorusso/master-degree-thesis}{Pruning
        layers of a neural network with a heuristic-based approach.}}

\cventry{}
    {University of Perugia, Perugia, IT}
    {Bachelor of Science in Computer Science (Networking) IT\@: 102/110 ---
        UK\@: 2:1}
    {\textbf{2006}}
    {}
    {Thesis: Wireless Broadband Network/WeConnect project}

\cventry{}
    {Ministry of Public Education, Commercial Technical Institute, Federico
        Cesi, Terni, IT}
    {Accountant programmer Diploma (Mercurio project) IT\@: 85/100 --- UK\@: A}
    {\textbf{2002}}
    {}
    {}

\section{Professional Training}

\cventry{Doulos}
    {Arm Ltd., Cambridge, UK}
    {Expert Product Development with Python}
    {\textbf{2022}}
    {}
    {}

\cventry{Learning Tree}
    {Arm Ltd., Cambridge, UK}
    {Advanced Python Training}
    {\textbf{2022}}
    {}
    {}

\cventry{Doulos}
    {Arm Ltd., Cambridge, UK}
    {C++ Programming for Embedded Systems}
    {\textbf{2021}}
    {}
    {}

\cventry{Doulos}
    {Arm Ltd., Cambridge, UK}
    {Practical Deep Learning}
    {\textbf{2020}}
    {}
    {}

\cventry{Udemy}
    {Online}
    {iPhone (iOS9) and Swift 2.2:
        \href{https://github.com/diegorusso/DipMatBeacon}{DipMatBeacon}}
    {\textbf{2016}}
    {University of Perugia project app for booking state of rooms}
    {}

\cventry{MongoDB}
    {Arm Ltd., Cambridge, UK}
    {MongoDB Administration and Developers training}
    {\textbf{2015}}
    {}
    {}

\section{Affiliations}
\cvitemwithcomment{Python Software Foundation}{Contributor (2023--2024),
    Triage Team (2024--2025), Core Developer (2025--present)}{2023--present}
\cvitemwithcomment{EuroPython Conference}{Organiser/Volunteer/Attendee,
    Florence, Berlin, Bilbao, Rimini, Edinburgh, Online}{2011 -- Present}

\section{Languages}
\cvlanguage{Italian}{Fluent}{Italian citizenship}
\cvlanguage{English}{C1}{British citizenship}
\cvlanguage{Spanish}{C1}{}

\begin{center}
Proudly written in \LaTeX\ and Vim: \url{https://github.com/diegorusso/cv}
\end{center}

\end{document}
