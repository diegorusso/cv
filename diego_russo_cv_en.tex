% possible options include font size ('10pt', '11pt' and '12pt'),
% paper size ('a4paper', 'letterpaper', 'a5paper', 'legalpaper',
% 'executivepaper' and 'landscape') and font family ('sans' and 'roman')
\documentclass[10pt,a4paper,sans]{moderncv}

% moderncv themes
% style options are 'casual' (default), 'classic', 'banking', 'oldstyle' and
% 'fancy'
\moderncvstyle{banking}
% color options 'black', 'blue' (default), 'burgundy', 'green', 'grey',
% 'orange', 'purple' and 'red'
\moderncvcolor{red}
% to set the default font; use '\sfdefault' for the default sans serif font,
% '\rmdefault' for the default roman one, or any tex font name
% \renewcommand{\familydefault}{\sfdefault}
% uncomment to suppress automatic page numbering for CVs longer than one page
% \nopagenumbers{}

% character encoding
% if you are not using xelatex or lualatex, replace by the encoding you are
% using
\usepackage[utf8]{inputenc}
% if you need to use CJK to typeset your resume in Chinese, Japanese or Korean
% \usepackage{CJKutf8}

% adjust the page margins
\usepackage[scale=0.92]{geometry}
% if you want to change the width of the column with the dates
\setlength{\hintscolumnwidth}{2.65cm}
% for the 'classic' style, if you want to force the width allocated to your
% name and avoid line breaks. be careful though, the length is normally
% calculated to avoid any overlap with your personal info; use this at your own
% typographical risks...
% \setlength{\makecvtitlenamewidth}{10cm}

% required when changes are made to page layout lengths
\AtBeginDocument{\recomputelengths}

% personal data
\name{Diego}{Russo}
% Below fields are optional, remove the line if not wanted
\title{Principal Software Engineer}
% {postcode city}{country}; the "postcode city" and "country" arguments can be
% omitted or provided empty
\email{me@diegor.it}
\phone[mobile]{+44 (0) 7428 251191}
\address{Cambridge, UK}
% the optional "type" of the phone can be "mobile" (default), "fixed" or "fax"
% \phone[fixed]{+2~(345)~678~901}
% \phone[fax]{+3~(456)~789~012}
\homepage{www.diegor.it}
\social[linkedin]{diegor}
\social[twitter]{diegor}
\social[github]{diegorusso}
\extrainfo{Updated on \today}
% '64pt' is the height the picture must be resized to, 0.4pt is the thickness
% of the frame around it (put it to 0pt for no frame) and 'picture' is the name
% of the picture file
%\photo[80pt][0.4pt]{images/diegor}
%\quote{``Now is better than never'' --- The Zen of Python}

% bibliography adjustments (only useful if you make citations in your resume,
% or print a list of publications using BibTeX)
% to show numerical labels in the bibliography (default is to show no labels)
%\makeatletter
%\renewcommand*
%    {\bibliographyitemlabel}
%    {\@biblabel{\arabic{enumiv}}}
%\makeatother
\renewcommand*{\bibliographyitemlabel}{[\arabic{enumiv}]}
% to redefine the bibliography heading string ("Publications")
% \renewcommand{\refname}{Articles}

% bibliography with mutiple entries
% \usepackage{multibib}
% \newcites{book,misc}{{Books},{Others}}

%Colour links in the CV
\AtBeginDocument{\hypersetup{colorlinks,urlcolor=red}}

% -----------------------------------------------------------------------------
%            content
% -----------------------------------------------------------------------------
\begin{document}
% -----       resume       ----------------------------------------------------
\maketitle
Proactive, innovative software development/engineering
professional with over 17 years’ expertise leading multiple global teams on
technical projects aimed at software/application improvement, system
enhancement, and process automation. Excellent cross-functional communicator
and motivational team leader with proven success delivering solutions in
alignment with business and stakeholder requirements. Highly motivated towards
applying diverse experience in machine learning, product-driven
environments as well as Open Source projects. Adept at leveraging knowledge
and experience in Python, software design, deployment, and technical support
to improve business processes and drive operational excellence.
Value-added skills in staff training, client/partner relationship management,
compliance/design standard management, troubleshooting, presentation, Python,
and Linux.

\section{Career Experience}
% arguments 3 to 6 can be left empty
% \cventry{year--year}{Degree}{Institution}{City}{\textit{Grade}}{Description}

%\cventry{}
%    {Position}
%    {Group, Company, Location, Country}
%    {\textbf{Period}}
%    {}
%    {\textbf{Technical Scope}: Python\newline
%    Stuff I did
%    \begin{itemize}
%        \item Stuff I did to highlight
%    \end{itemize}}

\cventry{}
    {Principal Software Engineer}
		{CE-SW Runtimes, Arm Ltd., Cambridge, UK}
    {\textbf{2023--present}}
    {}
    {\textbf{Technical Scope}: Tech Lead, CPython.\newline
		After a successful 6 months secondment, I’m joining the
		Runtimes team to continue the work I started with CPython.
		I will be engaging more with upstream community and bring my
		expertise to make CPython and its ecosystem work flawlessly
		for developers around the world on Arm platforms.}

\cventry{}
    {Principal Software Engineer}
		{CE-SW Runtimes (6 months secondment), Arm Ltd., Cambridge, UK}
    {\textbf{2023}}
    {}
    {\textbf{Technical Scope}: Tech Lead, CPython.\newline
		As part of the secondment programme, I spent six months in the CE-SW
		Runtimes team looking at the status of CPython (and its ecosystem)
		on Arm platforms.
		I've engaging with Arm's partners and the upstream community in
		order to address potential gaps.
    \begin{itemize}
			\item I presented at EuroPython 2023:
				\href{https://www.youtube.com/watch?v=nYf7r0lkTIs}{"Python on Arm architecture"}
    \end{itemize}}

\cventry{}
    {Principal Software Engineer/Staff Software Engineer}
    {ML Group, Arm Ltd., Cambridge, UK}
    {\textbf{2020--2023}}
    {}
    {\textbf{Technical Scope}: Tech Lead, Python, Machine Learning, IP
    Evaluation, Inference Advisor, Software Quality.\newline
    Act as lead for \href{https://pypi.org/project/mlia/}{Arm ML Inference
    Advisor} project while coordinating technical operations and communications
    between ML and Arm teams. Design, code, and debug software. Identify areas
    for software improvements and implement fixes.
    Train personnel when required and ensure maintenance of existing
    software. Provide recommendations on technologies to boost productivity.
    Oversee technical projects and address software-related queries. Manage
    software performance and attend team meetings. Formulate solutions to
    current and potential issues.
    Previous project: IP Selection Sandbox for ML\@.
    \begin{itemize}
        \item Help AI/ML developers to tailor and optimise their ML models to
            run well on Arm hardware: \url{https://pypi.org/project/mlia/}
        \item Enabled client’s opportunity to evaluate right IP for ML
            workloads.
        \item Established and grew Python Guild to over 700 members.
    \end{itemize}}

\cventry{}
    {Staff Software Engineer}
    {ISG, Arm Ltd., Cambridge, UK}
    {\textbf{2017--2019}}
    {}
    {\textbf{Technical Scope}: Jenkins/Pipelines, LAVA, Artifactory, Docker,
    Python, Bash, Linux, GitHub.\newline
    Led build, test, data representation, and validation automation
    infrastructure for company product. Liaised with multiple global teams and
    external partners. Monitored product development life cycle and outlined
    product requirements. Developed engineering system documents and design
    software. Provided detailed design specifications outlining functional
    requirements and architectural constraints. Implemented efficient coding
    practices and resolved software defects. Reviewed product specifications,
    system design, and protocol specifications. Reported on projects and
    issues. Contributed towards process improvement.
    \begin{itemize}
        \item Designed automated build and test infrastructure from scratch on
            real Cortex-A boards for custom Linux distribution.
    \end{itemize}}

\cventry{}
    {Staff Software Engineer}
    {DSG, Arm Ltd., Cambridge, UK}
    {\textbf{2013--2017}}
    {}
    {\textbf{Technical Scope}: Python, flask, bootstrap, AngularJS, MongoDB,
    Linux, svn/Git\newline
    Supervised and led eight-member team while managing Arm Compilerand OSS
    toolchains automation infrastructures. Ensured compliance with design
    principles while sharing knowledge, processes, and technologies with
    various teams. Hosted monthly forum to address stakeholder requirements.
    Acted as business representative on DSG infrastructures.
    \begin{itemize}
        \item Collaborated with GNU team on delivery of optimised GNU toolchain
            for Arm processors.
        \item Drove development of infrastructure for automated builds, tests,
            and benchmarks.
        \item Created application for triaging GCC test results, storing
            results in database in order for analysis and identification of
            root causes, patterns, trends, and automation of issue tracking
            using Jira.
    \end{itemize}}

\cventry{}
    {Senior Software Engineer}
    {Engineering IT, Arm Ltd., Cambridge, UK}
    {\textbf{2011--2013}}
    {}
    {\textbf{Technical Scope}: Python, MongoDB, RabbitMQ, Java, Jira, LSF,
    Perl, C, tcsh, bash, Linux, svn/Git\newline
    Contributed to multiple internal projects, including development of fault
    tolerant application which interacted with LFS cluster and AMQ server.
    Developed Jira plug-in aimed at interacting with internal software to
    enable synchronization of external and internal tickets. Maintained and
    enhanced internal IT systems.
    \begin{itemize}
        \item Planned and executed IT Early Career Scheme aimed at recruiting
            interns and graduates.
    \end{itemize}}

\cventry{}
    {Software/System Engineer}
    {R\&D, Forinicom Srl, Bastia Umbra, IT}
    {\textbf{2008--2011}}
    {}
    {\textbf{Technical Scope}: Python/Django, PostgreSQL, Debian, XEN, PyQT,
    Git\newline
    Delivered innovative product in collaboration with research and development
    team. Executed operations on embedded systems which included OS/software
    customisation for authentication management. Developed solution aimed at
    managing hotspots. Created captive portal for management of authentication,
    sign up, session logs, remote device signals, and integration with
    management software. Designed software for network monitoring.
    \begin{itemize}
        \item Set up internet connection and enabled fast connectivity for
            rural areas, providing services to hundreds of clients.
        \item Designed and implemented hotspot system with captive portal and
            credit card payments for numerous tourists; used at EuroPython
            2011.
    \end{itemize}}

\cventry{}
    {Python/Django Engineer}
    {Consorzio Miles, Servizi Integrati, Rome/Assisi, IT}
    {\textbf{2006--2008}}
    {}
    {\textbf{Technical Scope}: Python/Django, PostgreSQL, Linux\newline
    Delivered in collaboration with a team the replacement of paper-based
    manual processes with automated/digital processes. Maintained project in
    VB and led migration to Python/Django which enabled citizens to follow and
    update processes online and interact in real time with municipality.}

\subsection{Additional Experience}
%\cventry{}
%    {Position}
%    {Company, Location, Country}
%    {\textbf{Period}}
%    {\textbf{Technical Scope}: Python}
%    {}

\cventry{Remote, spare time project}
    {Software Engineer}
    {Opentaste Ltd., Global}
    {\textbf{2013--2015}}
    {\textbf{Technical Scope}: Technical advisor, code review, Python, Flask,
    MongoDB}
    {}

\cventry{Contract}
    {Objective-C Engineer}
    {Forinicom Srl., Bastia Umbra, IT}
    {\textbf{2011--05 / 2011--06}}
    {\textbf{Technical Scope}: Hotspot Captive Portal app, used at Europython
    2011 in Florence}
    {}

\cventry{Remote, Contract}
    {Python Engineer}
    {Exion Sagl, Manno, CH}
    {\textbf{2010--11/2011--01}}
    {\textbf{Technical Scope}: Python, Django, PostgreSQL}
    {}

\cventry{Remote, Contract}
    {Python Engineer}
    {Sauce Labs Inc., San Francisco, USA}
    {\textbf{2010--10/2011--01}}
    {\textbf{Technical Scope}: Python, Pylons, GitHub}
    {}

\cventry{Internship}
    {SEO Engineer}
    {Wedoit Sas., Assisi, IT}
    {\textbf{2005--11 / 2006--05}}
    {\textbf{Technical Scope}: SEO, Python, PHP}
    {}

\section{Education}

\cventry{}
    {University of Perugia, Perugia, IT}
    {Master of Science in Computer Science (Security) IT\@: 110/110 cum laude
        --- UK\@: First class honours}
    {\textbf{2021}}
    {}
    {Thesis: \href{https://github.com/diegorusso/master-degree-thesis}{Pruning
        layers of a neural network with a heuristic-based approach.}}

\cventry{}
    {University of Perugia, Perugia, IT}
    {Bachelor of Science in Computer Science (Networking) IT\@: 102/110 ---
        UK\@: 2:1}
    {\textbf{2006}}
    {}
    {Thesis: Wireless Broadband Network/WeConnect project}

\cventry{}
    {Ministry of Public Education, Commercial Technical Institute, Federico
        Cesi, Terni, IT}
    {Accountant programmer Diploma (Mercurio project) IT\@: 85/100 --- UK\@: A}
    {\textbf{2002}}
    {}
    {}

\section{Professional Training}

\cventry{Learning Tree}
    {Arm Ltd., Cambridge, UK}
    {Advanced Python Training}
    {\textbf{2022}}
    {}
    {}

\cventry{Doulos}
    {Arm Ltd., Cambridge, UK}
    {C++ Programming for Embedded Systems}
    {\textbf{2021}}
    {}
    {}

\cventry{Doulos}
    {Arm Ltd., Cambridge, UK}
    {Practical Deep Learning}
    {\textbf{2020}}
    {}
    {}

\cventry{Udemy}
    {University of Perugia, Perugia, IT}
    {iPhone (iOS9) and Swift 2.2:
        \href{https://github.com/diegorusso/DipMatBeacon}{DipMatBeacon}}
    {\textbf{2016}}
    {App used to check the booking state of university rooms}
    {}

\cventry{MongoDB}
    {Arm Ltd., Cambridge, UK}
    {MongoDB Administration and Developers training}
    {\textbf{2015}}
    {}
    {}

\cventry{Coursera}
    {Online}
    {\href{http://bit.ly/coursera-crypto1}{Cryptography 1}}
    {\textbf{2014}}
    {}
    {}

\cventry{ILX Group}
    {Arm Ltd., Cambridge, UK}
    {ITIL Foundation}
    {\textbf{2013}}
    {}
    {}

\cventry{}
    {University of Perugia, Perugia, IT}
    {iPhone (iOS4) and Objective-C}
    {\textbf{2011}}
    {Customised existent VOIP app}
    {}

\section{Licenses}

\cvline{}{Radio-amateur class A license (Nr. 020122/AN)/ International callsign
            IZ0OVB – C.I.S.A.R. Foligno’s section – 2007}

\section{Affiliations}
\cvitemwithcomment{Publication}{\href{https://doi.org/10.1109/ICCSA.2010.44}{
    The AES implementation based on OpenCL for multi/many core architecture}}
        {ICCSA 2010, Sangyo Uni., Fukuoka, JP}
\cvitemwithcomment{EuroPython Conference}{Volunteer/Attendee, Florence, Berlin,
    Bilbao, Rimini, Edinburgh, Online}{2011 – Present}

\section{Languages}
\cvlanguage{Italian}{Fluent}{Italian citizenship}
\cvlanguage{English}{C1}{British citizenship}
\cvlanguage{Spanish}{C1}{}

\begin{center}
Proudly written in \LaTeX\ and Vim: https://github.com/diegorusso/cv
\end{center}

\end{document}
