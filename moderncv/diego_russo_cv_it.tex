%% Copyright 2006-2010 Xavier Danaux (xdanaux@gmail.com).
%
% This work may be distributed and/or modified under the
% conditions of the LaTeX Project Public License version 1.3c,
% available at http://www.latex-project.org/lppl/.

\documentclass[10pt,a4paper]{moderncv}

% moderncv themes
\moderncvtheme[darkblue]{classic}                 % optional argument are 'blue' (default), 'orange', 'red', 'green', 'grey' and 'roman' (for roman fonts, instead of sans serif fonts)

% character encoding
\usepackage[utf8]{inputenc}                   % replace by the encoding you are using

% adjust the page margins
\usepackage[scale=0.85]{geometry}
\setlength{\hintscolumnwidth}{2.5cm}              % if you want to change the width of the column with the dates
% \AtBeginDocument{\setlength{\maketitlenamewidth}{6cm}}  % only for the classic theme, if you want to change the width of your name placeholder (to leave more space for your address details
\AtBeginDocument{\recomputelengths}                     % required when changes are made to page layout lengths

% personal data
\firstname{Diego}
\familyname{Russo}
\title{Sviluppatore Software}               % optional, remove the line if not wanted
\address{Cambridge}{Regno Unito}    % optional, remove the line if not wanted
\mobile{+44 7428 251191}                    % optional, remove the line if not wanted
\email{me@diegor.it}                      % optional, remove the line if not wanted
\homepage{http://www.diegor.co.uk}                % optional, remove the line if not wanted
\extrainfo{ultimo aggiornamento: Marzo 2014} % optional, remove the line if not wanted
\photo[100pt]{../images/diegor}                         % '64pt' is the height the picture must be resized to and 'picture' is the name of the picture file; optional, remove the line if not wanted
\quote{"Ora \`e meglio che mai" - Lo Zen del Python}                 % optional, remove the line if not wanted

% to show numerical labels in the bibliography; only useful if you make citations in your resume
\makeatletter
\renewcommand*{\bibliographyitemlabel}{\@biblabel{\arabic{enumiv}}}
\makeatother

% bibliography with mutiple entries
%\usepackage{multibib}
%\newcites{book,misc}{{Books},{Others}}

%\nopagenumbers{}                             % uncomment to suppress automatic page numbering for CVs longer than one page

\begin{document}
\maketitle

\section{Impiego ricercato}
\cvline{}{\large\textbf{Sono sempre alla ricerca di una posizione stimolante dove possa esprimere ed usare la mia passione per la programmazione e la tecnologia. Sviluppo quotidianamente in Python in ambienti *NIX. Essendo una persona dinamica, crescita professionale e personale sono molto importanti.}}

\section{Esperienza}
\subsection{Professionale}
\cventry{2011/10--Current Position}{Sviluppatore Senior in Engineering IT}{ARM Ltd, \url{http://www.arm.com/}}{Cambridge}{}{Lavorando in un team, sono coinvolto in molti progetti interni utilizzando \textbf{CentOS} e principalmente i seguenti linguaggi: \textbf{Python, Java, Perl, C, tcsh e bash}. Ho sviluppato da zero un affidabile e fault tolerant applicazione che interagisce con il cluster (\textbf {LSF}) e un server AMQ (\textbf {RabbitMQ}). Il linguaggio principale è stato Python utilizzando un database NoSQL (\textbf {MongoDB} configurato come \textit{ReplicaSet}). Ho anche sviluppato un plugin per Jira per interagire con un software interno per sincronizzare ticket esterni con quelli interni. Inoltre miglioro e correggo software interni utilizzando una vasta gamma di linguaggi. Ho anche esperienza con il cluster LSF, personalizzandone profondamente il suo comportamento, al fine di fornire una soluzione funzionale ai nostri clienti. Altri progetti minori sono legati a \textbf{SVN hooks}, FlexNet Server Manager, LSF monitor, applicazione web per visuallizare dati sui file system distribuiti. Insieme ad un mio collega, sono responsabile dell'IT ECS (\textit{Early Career Scheme}), gestendo tutte le fasi dalla selezione dei CV fino all'inizio carriera di neolaureati e stagisti.}

\cventry{12/2006--08/2008 09/2009--09/2011}{Programmatore Python/Django}{Consorzio Miles - Servizi Integrati, Roma \url{http://www.consorzio-miles.com/arianna/}}{Assisi}{}{Lavorando in un team, ho sviluppato un'applicazione gestionale per il comune di Bettona utilizzando Django, Python, PostgreSQL, Linux, Apache, per \textbf{l'informatizzazione dei servizi}, per la gestione delle anagrafiche nonch\'e delle pratiche edilizie ed urbanistiche e del calcolo della tassa ICI con aggiornamenti dei dati catastali. Inoltre ho creato un'avanzata interfaccia web per la presentazione di proposte di pratiche, conferenza dei servizi on-line, integrazione di procedimenti, visione di mappe catastali in \textbf{DXF} e produzione di stampe personalizzate ed automatizzate. Durante il progetto ho utilizzato controlli di versione del software (SVN/GIT), con relativa interfaccia web (trac) per la gestione dei ticket.}

\cventry{05/2011--06/2011}{Programmatore Objective-C}{Forinicom Srl, Bastia Umbra, \url{http://www.forinicom.it}}{Assisi}{}{Sviluppata un'applicazione per iPhone che ti permette di effettuare l'auto-login negli hotspot ComCom (\url{http://www.com-com.it/}). Questa applicazione \'e usata dai partecipanti della conferenza Europython 2011.}

\cventry{04/2008--02/2011}{Programmatore e sistemista nel reparto Ricerca e Sviluppo}{Forinicom Srl \url{http://www.forinicom.it}}{Bastia Umbra}{}{Lavorando in un team di ricerca e sviluppo per la creazione di un prodotto innovativo ed unico nel mercato delle comunicazioni senza fili (WiFi), ho lavorato in un primo periodo su \textbf{dispositivi embedded} (ubnt, alix, pcengines) personalizzando fortemente il sistema operativo (ubnt, openwrt) ed i software per gestire l'autenticazione (hostapd, wpa-supplicant). Dopo questa prima fase mi sono concentrato sullo sviluppo di software per il \emph{flashing} di tali dispositivi e per la produzione su larga scala. Abbiamo inoltre sviluppato una soluzione completa per la gestione di \textbf{un sistema di hotspot}: mi sono occupato dello sviluppo lato server in modo da gestire autenticazioni, log delle sessioni, registrazioni, gestione dei segnali dai nodi, integrazione con i nostri gestionali, pagamenti con carta di credito ed autenticazione tramite SMS, il tutto in regola con la normativa Pisanu. Come ultimo incarico ho creato un software per il monitoring della rete. Questo tratta di un'applicazione \textbf{stand-alone in PyQT}, utilizzando delle API interne basate su Django. Le tecnologie utilizzate sono per la maggiore Python/Django con database PostgreSQL su sistemi Debian virtualizzati su XEN.}

\cventry{11/2010--01/2011}{Programmatore Python/Django}{Exion Sagl, Manno, Svizzera, \url{http://www.exion.ch/}}{Assisi, da remoto}{}{Completamento di una \textbf{WebTV per adulti} interamente sviluppato in Python/Django con database PostgreSQL su piattaforma Linux/Apache e backend di streaming in Red5. Il lavoro \'e interamente gestito in autonomia utilizzando GIT come software di revisione del software.}

\cventry{10/2010--01/2011}{Programmatore Python/Pylons}{Sauce Labs Inc, San Francisco, California, USA, \url{http://saucelabs.com/}}{Assisi, da remoto}{}{Lavorando da remoto, implementazioni di nuove funzionalit\'a, correzione di bug, modifiche strutturali al sito della Sauce Labs. Portale scritto in Python/Pylons utilizzando \url{github.com} per la revisione del codice.}

\subsection{Miscellanea}
\cventry{06/2011}{Insegnamento - Corso di computer avanzato}{Centro Studi Citt\'a di Foligno, \url{http://www.cstudifoligno.it/}}{}{}{Insegnato ad una classe di 10 persone l'esistenza del mondo open source, installando software open source su Windows e poi procedere all'installazione sui propri portatili.}

\cventry{01/2011--06/2011}{Stage - sviluppo iPhone ed iPad}{Universit\'a degli studi di Perugia, Dipartimento di Informatica, \url{http://informatica.unipg.it}}{Assisi}{}{Seguendo le lezioni della Stanford University, ho imparato di pi\'u il mondo Objective-C ed iPhone, sviluppando piccole applicazioni. Come progetto finale ho personalizzato un'applicazione VOIP per iPhone, basata su Linphone (\url{http://www.linphone.org/}).}

\cventry{11/2005--05/2006}{Stage - S.E.O. Search Engine Optimization}{WEDOIT sas, \url{http://www.wedoit.us}}{Assisi}{}{Lavorando in un team ho acquisito conoscenze di S.E.O. e dei suoi meccanismi. Lo stage prevedeva l'ottimizzazione S.E.O. di un insieme di siti utilizzando tecniche di \emph{pageranking} e \emph{link popularity}. Inoltre mi sono occupato dell'amministrazione di un server virtuale (basato su Debian) e dello sviluppo di applicazione in Python e PHP orientate al S.E.O.}

\cventry{02/2002}{Stage abbinato al progetto IFS, Impresa Formativa Simulata}{IOSA CARLO Srl, \url{http://www.iosacarlo.com}}{Terni}{}{Durante lo stage ho gestito della rete interna dell'impresa}

\section{Istruzione e formazione}
\cventry{Dal 10/2010}{Specializzazione di Informatica, indirizzo di ''Sicurezza Informatica''}{Universit\'a degli studi di Perugia, Dipartimento di Informatica, \url{http://informatica.unipg.it}}{}{\textit{Inscritto}}{Sostenuti i seguenti esami con eccellenti voti: Simulazione, Programmazione Avanzata e laboratorio, Sistemi operativi avanzati e laboratorio, Informatica Teorica, Sicurezza, Basi di dati avanzati e data mining, Diritto dell'informazione.}

\cventry{04-2013--06-2013}{Corso di Inglese Avanzato}{Sixth Form College}{Cambridge, UK}{\textbf{livello C1-C2}}{Competenze nel \textbf{Quadro comune europeo di riferimento per la conoscenza delle lingue} (\url{http://it.wikipedia.org/wiki/Quadro_comune_europeo_di_riferimento_per_la_conoscenza_delle_lingue})}

\cventry{05/2012--06/2012 10-2012--11-2012 01-2013--02-2013}{Corso di Portoghese Brasiliano}{Sixth Form College}{Cambridge, UK}{\textbf{livello A2}}{Competenze nel \textbf{Quadro comune europeo di riferimento per la conoscenza delle lingue} (\url{http://it.wikipedia.org/wiki/Quadro_comune_europeo_di_riferimento_per_la_conoscenza_delle_lingue})}

\cventry{10/2010--05/2011}{Corso di Inglese}{Istituto comprensivo ''Volumnio'' Ponte San Giovanni}{Perugia}{\textbf{livello B1}}{Competenze nel \textbf{Quadro comune europeo di riferimento per la conoscenza delle lingue} (\url{http://it.wikipedia.org/wiki/Quadro_comune_europeo_di_riferimento_per_la_conoscenza_delle_lingue})}

\cventry{10/2009--05/2010}{Corso di Spagnolo}{Istituto comprensivo ''Volumnio'' Ponte San Giovanni}{Perugia}{\textbf{livello B1}}{Competenze nel \textbf{Quadro comune europeo di riferimento per la conoscenza delle lingue} (\url{http://it.wikipedia.org/wiki/Quadro_comune_europeo_di_riferimento_per_la_conoscenza_delle_lingue})}

\cventry{08/2009--03/2010}{Pubblicazione del paper \cite{aes}}{Universit\'a degli studi di Perugia, Dipartimento di Informatica \url{http://informatica.unipg.it}}{}{}{Preparazione e pubblicazione del paper ''The AES implentation based on OpenCL for multi/many core architecture'' per l'annuale conferenza ICCSA 2010 (\url{www.iccsa.org}) alla Sangyo University, Fukuoka in Giappone. Il paper tratta di un' implementazione di AES eseguito su core GPU NVIDIA/ATI.}

\cventry{02/2007--07/2007}{Patente di operatore di stazione di \textbf{radioamatore di classe A}}{C.I.S.A.R. Sezione di Foligno}{}{IDONEO, Nominativo internazionale \textbf{IZ0OVB}}{Durante il corso per aspiranti radioamatori ho acquisito ottime conoscenze di radiotecnica, apparecchiature radio e loro funzionamento. Inoltre non sono mancati cenni di fisica e chimica (magnetismo, elettromagnetismo)}

\cventry{03/2007}{Corso di Spagnolo}{Inhispania Intlance S.L \url{http://www.inhispania.com/}}{Madrid, Spagna}{\textbf{Livello A2}}{Durante il periodo trascorso a Madrid, in questa scuola ho approfondito conoscenze aggiuntive riguardo la grammatica di base e la cultura generale spagnola.}

\cventry{12/2006}{Corso sulle certificazioni ISO}{WEDOIT sas, \url{http://www.wedoit.us}}{Assisi}{}{Corso di formazione sulla sicurezza e certificazioni ISO riguardante ISO 27001:2005, politica per la sicurezza delle informazioni, analisi dei rischi (RA), analisi dei controlli della ISO 17799:2005, trattamento dei rischi (RTP), processo di certificazione, panorama delle certificazioni per gli audit, piano di audit e checklist, rapporto di audit, sguardo alle future certificazioni}

\cventry{10/2002--11/2006}{Laurea triennale in Informatica}{Universit\'a degli studi di Perugia, Dipartimento di Informatica, \url{http://informatica.unipg.it}}{}{\textbf{102/110}}{Laurea triennale in informatica, \textbf{indirizzo ''Reti di computer''}: Matematica (analitica e discreta), Programmazione (C, Java, Php, html, xml, xsl, dtd, Pascal, scripting bash e csh, VB.NET, VRML), Database (Mysql, MS Access e loro interazioni con linguaggi di programmazione), Reti (ATM, xDSL, Mpls, X.25, Frame Relay) tipologie (wireless, wired) e loro interazioni, Conoscenza di sistemi multimediali, Cenni di calcolo parallelo (mpi)}

\cventry{09/1996--06/2002}{Diploma in ragioniere programmatore (progetto Mercurio)}{Ministero della Pubblica Istruzione - I.T.C. ''Federico Cesi''}{Terni}{\textbf{85/100}}{Le materie definite dal Ministero dell'Istruzione e previste dal percorso di studio dell'Istituto Tecnico Commerciale sono: Scienze della Materia, Matematica e Laboratorio, Scienze della Natura, Trattamento Testi e Dati, Seconda lingua straniera (Francese), Diritto ed Economia, Economia Aziendale, Economia Politica e Scienza delle Finanze, Lingua e letteratura italiana, Storia, Informatica Gestionale, Matematica applicata, Prima lingua straniera (Inglese), Diritto.}

\cventry{2001--2002}{Progetto Nazionale IFS (Impresa Formativa Simulata)}{Ministero della Pubblica Istruzione - I.T.C. ''Federico Cesi''}{Terni}{Certificate of attendance}{Simulazione di un'impresa di smaltimento rifiuti, affiancati dall'impresa Iosa Carlo S.r.l. (\url{http://www.iosacarlo.com}). Nell'ambito del progetto ho coordinato il lavoro di tutti gli studenti, realizzando l'organigramma dell'azienda simulata e sviluppando il sito dell'azienda.}

\section{Tesi di Laurea}
\cvline{title}{\textbf{Wireless Broadband Network - progetto WeConnect} (07/2006--12/2006)}
\cvline{supervisors}{Simone Brunozzi, Sergio Tasso}
\cvline{description}{\small Il lavoro di tesi consisteva nello sviluppare una rete WiFi in grado di coprire zone in \textbf{digital-divide}. Grazie a questo progetto ho acquisito ampia conoscenza delle reti wireless, della normativa che ne regola il funzionamento, del sistema operativo RouterOS (\url{www.mikrotik.com}), del protocollo AAA e del server FreeRADIUS. Infine ho amministrato server per l'erogazione di vari servizi di rete: mail (Postfix), server web (Apache), DNS (pdns), firewall (iptables), database (PostgreSQL), hotspot (Chillispot), OS Debian, Voyage (OS per sistemi embedded, basata su Debian).}

\section{Lingue}
\cvlanguage{Italian}{\textbf{Madre Lingua}}{}
\cvlanguage{English}{\textbf{livello C1}}{\textbf{Preliminary English Test} (PET), 05/2011}
\cvlanguage{Spanish}{\textbf{livello C1}}{\textbf{Diploma de Español como Lengua Extranjera} (D.E.L.E.), 05/2010}
\cvlanguage{Portoghese (BR)}{\textbf{livello A2}}{}

\section{Conoscenze Informatiche}
\cvline{Programmazione, Scripting, Linguaggi di Markup}{{\huge Python}, {\small sh}, {\large Javascript}, {\tiny CSS}, {\large bash}, {\large HTML}, {\huge Perl}, {\tiny XML}, {\large SQL}, {\large JSON}, {\normalsize LSL (Linden Scripting Language)}, {\huge Java}, {\normalsize C}, {\large Objective-C}, {\small PHP}, {\normalsize LaTeX}} 
\cvline{Framework}{Django, Flask, JQuery, Nokia Qt4, Pylons}
\cvline{Sistemi Operativi}{Linux (Debian based), Unix, OSX, XEN e virtualizzazione, OpenWRT, Ubnt (\url{http://www.ubnt.com/}), Microsoft Windows}
\cvline{IDE}{Vim (Non \'e un vero IDE), TextMate, XCode, Eclipse}
\cvline{Database}{PostgrSQL, MongoDB, MySQL, SQLite, CouchDB}

\section{Interessi}
\cvline{Lingue}{\small Ho imparato Spagnolo come autodidatta. Al momento parlo Inglese, Spagnolo, Italiano e Portoghese. Il mio obbiettivo personal: \textbf{5 lingue entro il 2015}. Ho anche interesse per l'esperanto}
\cvline{Tecnologia}{\small Attratto da qualsiasi cosa abbia un processore}
\cvline{Fotografia}{\small Foto amatoriale, mi diverto con una reflex}
\cvline{Musica}{\small Livello hobbistico. Ho suonato pianoforte e chitarra e mi piace ascoltare qualsiasi tipo di musica, dalla salsa al metal}
\cvline{Studi}{\small Propenso all'apprendimento ed allo studio}
\cvline{Scienze}{\small Attrazione per le materie scientifiche in generale}
\cvline{Curioso}{\small Questo è come mi definisco}
\cvline{Sport}{\small Salsa cubana, Squash. In passato: Capoeira, Kungfu, Nuoto}
\cvline{Puzzle}{\small Amo risolvere ogni tipo di puzzle ed ho la passione per i cubi di Rubik: dimensioni risolte sono 2x2x2, 3x3x3, 4x4x4, 5x5x5 and 9x9x9}


\section{Patente/i}
\cvlistitem{Patente di Guida B}
\cvlistitem{Patente di Operatore di stazione di radioamatore di classe A (nr. 020122/AN), nominativo Internazionale \textbf{IZ0OVB}}

\section{Informazioni Extra}
\cvlistitem{Mambro BCS (\url{http://www.bcs.org})}
\cvlistitem{in regola con gli obblighi di leva (rinvio per studio)}
\cvlistitem{Linux Registered User \#399008}
\cvlistitem{socio ordinario e donatore dell'AVIS (Associazione Volontari Italiani Sangue)}
\cvlistitem{stato civile: celibe}

% Publications from a BibTeX file without multibib\renewcommand*{\bibliographyitemlabel}{\@biblabel{\arabic{enumiv}}}% for BibTeX numerical labels
\nocite{*}
\bibliographystyle{plain}
\bibliography{publications}       % 'publications' is the name of a BibTeX file

\section{About me}
\cvline{}{Vista \textbf{la mia passione per l'informatica} ho sviluppato nell'arco degli anni una serie di competenze che variano in molti settori della stessa.\newline
Sin dagli anni degli studi superiori, oltre la buona rendita scolastica, ho creato e mantenuto un'attivit\'a extra-curriculare \textbf{al di sopra della media}: tra le varie iniziative a cui ho partecipato ricordo il ''Corso di base sulla multimedialità'', ''Exposcuola 2000 a Paestum'', ''Corso di informatica di base in funzione di tutor'', ''Corso di alfabetizzazione di computer di base in funzione di tutor a persone con et\'a superiore a 65 anni'', ''XI Settimana della cultura scientifica e tecnologica'', ''Pluto Meeting 2001'' e ''Attivit\'a di tutor/referente di un gruppo di altri 6 studenti/tutor, per le attivit\'a di POTENZIAMENTO DI ITALIANO delle prime classi, in ambito del progetto ''Accoglienza, Recupero, Potenziamento nelle Prime Classi''''. In tutti i progetti menzionati, ho partecipato in maniera attiva dedicando tempo e volont\'a nell'apprendere cose nuove riguardante le nuove tecnologie informatiche e non.\newline
Dal mio primo computer, ho avuto una certa passione per il mondo \textbf{open source} e tutto quello che lo riguarda: infatti ho amministrato macchine \textbf{Linux} con varie distribuzioni, come RedHat 7.3, Slackware 7.1 fino ad arrivare a macchine Debian (dalla versione 3.0 a quelle attuali). Tramite questa esperienza ho maturato una certa abilit\'a e conoscenza nella gestione di macchine Linux: scripting bash, configurazione e compilazione del kernel, servizi di rete, patch per il kernel, linguaggio C. Oltre a Linux uso \textbf{OSX} per l'utilizzo quotidiano. Visto il continuo utilizzo e la mia passione per l'informatica ho approfondito lo studio di quest'ultimo.\newline
Ho partecipato attivamente come contributore alla scrittura della guida \url{http://www.ubuntusemplice.org/} (versione 6.06 e 7.10). In questo progetto sono stato autore e reviewer di vari capitoli, ho amministrato le macchine che ospitavano il sito, il wiki, il blog e la mailing list.\newline
\textbf{Inoltre ho una grande passione per quanto riguarda la programmazione}: conosco molti linguaggi anche in ambiti diversi tra di loro come Python, C, PHP, java, LSL (Linden Scripting Language). L'LSL l'ho studiato durante la mia attivit\'a su \textbf{Second Life}: infatti ho collaborato su molti progetti italiani presenti nel metaverso come Assisi \url{http://www.secundavita.it}, Milano e Marostica del progetto \textbf{Italia Vera}.\newline
Ho una buona conoscenza di applicazioni grafiche (Gimp, Photoshop) e di strumenti per l'ufficio come Openoffice.org ed iWork (per OSX)\newline
Dotato di buona determinazione riesco a lavorare sia in un team, organizzandomi con i colleghi, sia individualmente gestendo in piena autonomia tutto il flusso di lavoro. Abituato a lavorare in team, ho un rapporto costruttivo e collaborativo con le persone che mi circondano, quali colleghi e collaboratori. Sono una persona socievole, simpatica e con buone doti comunicative; il mio sito \'e fonte di contatti e scambi sociali continui con altre persone tecniche e meno techiche.}
\end{document}
