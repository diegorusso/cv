%% Copyright 2006-2010 Xavier Danaux (xdanaux@gmail.com).
%
% This work may be distributed and/or modified under the
% conditions of the LaTeX Project Public License version 1.3c,
% available at http://www.latex-project.org/lppl/.

\documentclass[10pt,a4paper]{moderncv}

% moderncv themes
\moderncvtheme[darkred]{classic}                 % optional argument are 'blue' (default), 'orange', 'red', 'green', 'grey' and 'roman' (for roman fonts, instead of sans serif fonts)

% character encoding
\usepackage[utf8]{inputenc}                   % replace by the encoding you are using

% adjust the page margins
\usepackage[scale=0.85]{geometry}
\setlength{\hintscolumnwidth}{2.5cm}              % if you want to change the width of the column with the dates
% \AtBeginDocument{\setlength{\maketitlenamewidth}{6cm}}  % only for the classic theme, if you want to change the width of your name placeholder (to leave more space for your address details
\AtBeginDocument{\recomputelengths}                     % required when changes are made to page layout lengths

% personal data
\firstname{Diego}
\familyname{Russo}
\title{Software Developer}               % optional, remove the line if not wanted
\address{Via Ruggero Vallemani, 14}{06086, Petrignano (PG)}    % optional, remove the line if not wanted
\mobile{+39 334 5873886}                    % optional, remove the line if not wanted
\phone{+39 0744 930614}                      % optional, remove the line if not wanted
\email{me@diegor.it}                      % optional, remove the line if not wanted
\homepage{http://www.diegor.it}                % optional, remove the line if not wanted
\extrainfo{last update: June 2011} % optional, remove the line if not wanted
\photo[100pt]{../images/diegor}                         % '64pt' is the height the picture must be resized to and 'picture' is the name of the picture file; optional, remove the line if not wanted
\quote{"Now is better than never" - The Zen of Python}                 % optional, remove the line if not wanted

% to show numerical labels in the bibliography; only useful if you make citations in your resume
\makeatletter
\renewcommand*{\bibliographyitemlabel}{\@biblabel{\arabic{enumiv}}}
\makeatother

% bibliography with mutiple entries
%\usepackage{multibib}
%\newcites{book,misc}{{Books},{Others}}

%\nopagenumbers{}                             % uncomment to suppress automatic page numbering for CVs longer than one page

\begin{document}
\maketitle

\section{Desired employment}
\cvline{}{\large\textbf{I am looking for a position where I can express and use my passion for programming and technology. I mostly develop in Python/Django using OSX and Linux daily. Available for positions that give me personal and professional growth.}}

\section{Experience}
\subsection{Vocational}
\cventry{2006/12--2008/08 2009/09--Present}{Python/Django Programmer}{Consorzio Miles - Servizi Integrati, Rome \url{http://www.consorzio-miles.com/arianna/}}{Assisi}{}{Working in a team, I developed a management application for the municipality of Bettona using Django, Python, PostgreSQL, Linux, Apache, for the \textbf{computerization of services}, the management of personal data, building practices, urban planning, calculation of ICI tax and updating of land registry data. Also I created an advanced web interface for sending proposed practices, on-line services conference, integration process, exploration of cadastral map in \textbf{DXF} and production of customized an automated printing. During the project I used revision control systems (SVN/GIT) with related web interface (trac) to manage tickets.}

\cventry{2011/05--2011/06}{Objective-C Programmer}{Forinicom Srl, Bastia Umbra, \url{http://www.forinicom.it}}{Assisi}{}{Developed an iPhone application that permit you to auto-login into ComCom Hotspot (\url{http://www.com-com.it/}). This application is used by the attendees in the Euroython Conference 2011 in Florence.}

\cventry{2008/04--2011/02}{Programmer and System Engineer in Research and Development Department}{Forinicom Srl, \url{http://www.forinicom.it}}{Bastia Umbra}{}{Working in a research and development team to create of a innovative and unique product in the wireless communications market (WiFi), I initially worked on \textbf{embedded systems} (ubnt, alix, pcengines), customizing the operating system (ubnt, openwrt) and the softwares to manage authenitcation (hostapd, wpa-supplicant). After this first phase, I focused on the of software \textbf{to flash} these devices and on large-scale production software. Also, we developed a complete solution for managing \textbf{hotspots}: I worked on server-side development to manage authentication, sessions log, signups, signals management from remote devices, integration with our management software, payment via credit card and authentication via SMS, complying with Pisanu law. My final task was to create software for network monitoring. It is a \textbf{PyQT stand-alone} application, using internal django based API. The technologies mostly used are Python/Django with PostgreSQL database on Debian OS virtualized on XEN}

\cventry{2010/11--2011/01}{Python/Django Programmer}{Exion Sagl, Manno, Switzerland, \url{http://www.exion.ch/}}{Assisi, remotely}{}{Setting up an \textbf{Adult WebTV} entirely developed in Python/Django with PostgreSQL database on Linux/Apache platform and Red5 as streaming server. The work is managed independently using GIT as revision control system.}

\cventry{2010/10--2011/01}{Python/Pylons Programmer}{Sauce Labs Inc, San Francisco, California, USA, \url{http://saucelabs.com/}}{Assisi, remotely}{}{Implementing new features, bug fixing, structural changes to the site of Sauce Labs. Distance work coordination. Site is developed in Python/Pylons using \url{github.com} as revision control system platform.}

\subsection{Miscellaneous}
\cventry{2011/06}{Teaching - Advanced computer course}{Centro Studi Citt\'a di Foligno, \url{http://www.cstudifoligno.it/}}{}{}{Taught a class of 10 people the existence of the open source world, installing open source software on Windows and then proceed to install Ubuntu on their laptop.}

\cventry{2011/01--2011/06}{Trainee - iPhone and iPad development}{University of Perugia, Computer Science department, \url{http://informatica.unipg.it}}{Assisi}{}{Following lessons of Stanford University, I trained myself to Objective-C and iPhone world, developing small applications. As final project I customized a VOIP application for iPhone based on Linphone (\url{http://www.linphone.org/}).}

\cventry{2005/11--2006/05}{Trainee - S.E.O. Search Engine Optimization}{WEDOIT sas, \url{http://www.wedoit.us}}{Assisi}{}{Working in a team I acquired knowledge of S.E.O. and its behavior. The internship included S.E.O. optimization of various websites, using \emph{pagerank} and \emph{link popularity} methods. Also I worked as a system engineer of Debian-based virtualized server and I developed a S.E.O. oriented application in Python and PHP.}

\cventry{2002/02}{Trainee combined with IFS project, Impresa Formativa Simulata (Enterprise Training Simulation)}{IOSA CARLO Srl, \url{http://www.iosacarlo.com}}{Terni}{}{Administration of enterprise network}

\section{Education and Training}
\cventry{Since 2010/10}{Specialization course, with mayor in Computer Science, ``Security''}{University of Perugia, Computer Science department, \url{http://informatica.unipg.it}}{}{\textit{Enrolled}}{Passed following exams with excellent mark: Simulation, Advanced programming, Advanced Operating Systems, Advanced Operating Systems Laboratory.}

\cventry{2010/10--2011/05}{English Course}{Comprehensive School ``Volumnio'' Ponte San Giovanni}{Perugia}{\textbf{B1 level}}{The skills achieved are all included in \textbf{Common European Framework of Reference for Languages} (\url{http://en.wikipedia.org/wiki/Common_European_Framework_of_Reference_for_Languages})}

\cventry{2009/10--2010/05}{Spanish Course}{Comprehensive School ``Volumnio'' Ponte San Giovanni}{Perugia}{\textbf{B1 level}}{The skills achieved are all included in \textbf{Common European Framework of Reference for Languages} (\url{http://en.wikipedia.org/wiki/Common_European_Framework_of_Reference_for_Languages})}

\cventry{2009/08--2010/03}{Paper publication \cite{aes}}{University of Perugia, Computer Science department, \url{http://informatica.unipg.it}}{}{}{Preparation and publication of ``The AES implementation based on OpenCL for multi/many core architecture'' paper for the yearly conference ICCSA 2010 (\url{www.iccsa.org}) at Sangyo University, Fukuoka in Japan. The paper discuss the implementation of an AES algorithm that runs on NVIDIA/ATI graphics card.}

\cventry{2007/02--2007/07}{Radio-amateur \textbf{class A license}}{C.I.S.A.R. Foligno's section}{}{PASSED, International Callsign \textbf{IZ0OVB}}{During the course for radio-amateur i acquired excellent knowledge of radio technology basics, radio devices and its usage and basics of Physics and Chemistry (magnetism, elettromagnetism)}

\cventry{2007/03}{Spanish Course}{Inhispania Intlance S.L, \url{ http://www.inhispania.com/}}{Madrid, Spain}{\textbf{A2 Level}}{During the time in Madrid, I studied Spanish grammar and general Spanish culture.}

\cventry{2006/12}{ISO certifications course}{WEDOIT sas, \url{http://www.wedoit.us}}{Assisi}{}{Training course on security and ISO certifications, covering ISO 27001:2005, policy for Information Security, Risks Analysis (RA), analysis of controls of ISO 17799:2005, Risk Transfer Process (RTP), certification process, overview of existing certification audits, audit plan and checklist, audit report, a look at future certifications.}

\cventry{2002/10--2006/11}{Bachelor Degree in Computer Science}{University of Perugia, Computer Science department, \url{http://informatica.unipg.it}}{}{\textbf{102/110}}{Computer science Bachelor Degree, \textbf{mayor ``Network''}: Mathematics (analytical and discrete), Programming (C, Java, Php, html, xml, xsl, dtd, Pascal, scripting bash and csh, VB.NET, VRML), Databases (Mysql, MS Access and related programming language), Networks (ATM, xDSL, Mpls, X.25, Frame Relay) types (wireless, wired) and interaction between them, Knowledge of multimedia system, overview of parallel computing (mpi)}

\cventry{1996/09--2002/06}{Accountant programmer Diploma (Mercurio project)}{Ministry of Public Education - I.T.C. (Commercial technical institute) ``Federico Cesi''}{Terni}{\textbf{85/100}}{Matters covered by the course (Commercial technical institute) of study as defined by the Ministry of Public Education: Chemistry / Physics, Mathematics and Computer Laboratory, Natural Science, Word processing and data, Second foreign language (French), Law and Economics, Business, Economics and Financial Science, Italian Language and Literature, History, Computer Management, Applied Mathematics, First foreign language (English), Law.}

\cventry{2001--2002}{National Project IFS (Enterprise Training Simulation}{Ministry of Public Education - I.T.C. (Commercial technical institute) ``Federico Cesi''}{Terni}{Certificate of attendance}{Simulation of waste disposal firm, backed by Iosa Carlo S.r.l. (\url{http://www.iosacarlo.com}).
Within the project I coordinated the work of all students, building the simulated organization chart and programming the website.}

\section{Bachelor Thesis}
\cvline{title}{\emph{Wireless Broadband Network - Weconnect project} (2006/07--2006/12)}
\cvline{supervisors}{Simone Brunozzi, Sergio Tasso}
\cvline{description}{\small The thesis was to develop a WiFi network in order to coverage \textbf{digital-divide} areas. Thanks to this project, I acquired a wide knowledge about WiFi networks and their behavior, legislation that governs the operation, RouterOS operating system (\url{www.mikrotik.com}), AAA protocol and Radius server. Finally I administered a server for the provision of various network services: mail (Postfix), web server (Apache), DNS (pdns), firewall (iptables), database (PostgreSQL), hotspot (Chillispot), Debian OS, Voyage (OS for embedded Debian based system).}

\section{Languages}
\cvlanguage{Italian}{\textbf{Mother tongue}}{}
\cvlanguage{English}{\textbf{B1 level}}{\textbf{Preliminary English Test} (PET), 2011/05}
\cvlanguage{Spanish}{\textbf{B1 level}}{\textbf{Diploma de Español como Lengua Extranjera} (D.E.L.E.), 2010/05}

\section{Computer skills}
\cvline{Programming, Scripting, Markup Languages}{ {\huge Python}, {\small sh}, {\huge Javascript}, {\tiny CSS}, {\large bash}, {\scriptsize JSON}, {\huge HTML}, {\tiny XML}, {\large SQL}, {\normalsize LSL (Linden Scripting Language)}, {\footnotesize C}, {\huge Objective-C}, {\small PHP}, {\normalsize LaTeX}}
\cvline{Frameworks}{Django, JQuery, Nokia Qt4, Pylons}
\cvline{Operating Systems}{Linux (Debian based), Unix, OSX, XEN and Virtualization, OpenWRT, Ubnt (\url{http://www.ubnt.com/}), Microsoft Windows}
\cvline{IDEs}{TextMate, Vim, XCode, Eclipse}
\cvline{Databases}{PostgrSQL, MySQL, SQLite, CouchDB}

\section{Interests}
\cvline{Languages}{\small Learning the Spanish language by myself. I would like to learn another language.}
\cvline{Technology}{\small Attracted by all things that have a processor}
\cvline{Photos}{\small Amateur photos}
\cvline{Music}{\small Hobby level. I played piano and guitar.}
\cvline{Studies}{\small Willing to learn and study}
\cvline{Science}{\small Attraction for science in general}
\cvline{Curious}{\small That's how I define myself}

\section{Licence(s)}
\cvlistitem{Driving license type B.}
\cvlistitem{Operator license for amateur radio station class A (nr. 020122/AN), International callsign \textbf{IZ0OVB}}

\section{Extra Information}
\cvlistitem{comply with compulsory military service (referring to studies)}
\cvlistitem{Linux Registered User \#399008}
\cvlistitem{ordinary member of the LUG Perugia}
\cvlistitem{AVIS (Blood Donor Italian Association) ordinary member}
\cvlistitem{marital status: single}

% Publications from a BibTeX file without multibib\renewcommand*{\bibliographyitemlabel}{\@biblabel{\arabic{enumiv}}}% for BibTeX numerical labels
\nocite{*}
\bibliographystyle{plain}
\bibliography{publications}       % 'publications' is the name of a BibTeX file

\section{About me}
\cvline{}{Because of \textbf{my passion for computing}, I have developed over the years wide range of skills many areas computer-related.\newline
During the years of higher education, beyond academic success, I created and maintained an \textbf{above average} extra-curricular activity: among other initiatives in which I participated, I remember ''The basic course on multimedia'',''Exposcuola 2000 in Paestum'', ''Basic computer as tutor role'', ''Computer literacy course as tutor to senior people (over 65 years old)'', ''XI week of scientific and technological culture'', ''Pluto Meeting 2001'' and ''Main tutor of a group of 6 people for the course of IMPROVING ITALIAN for new students in the project ''Reception, Recovery, Empowerment in early grades''''. In all the projects mentioned, I participated actively devoting time and effort in learning new things about new information technology and other subjects.\newline
From the moment I had my first computer, I have had a certain passion for the \textbf{open source} world and all that concerns it: in fact I have managed machines with \textbf{Linux} distributions such as RedHat 7.3, Slackware 7.1 up to machines to Debian (from version 3.0 to current ones). Through this experience I have gained skill and knowledge in the management of Linux: bash scripting, configuring and compiling the kernel, network services, patching the kernel, C language. In addition to Linux, I daily use OSX.\newline
I was active as a contributor to the writing of the guide \url{http://www.ubuntusemplice.org/} (version 6.06 and 7.10). On this project I was the reviewer and author of several chapters, I administered the machine that hosts the site, wiki, blog and mailing list.\newline
\textbf{I also have a great passion for programming languages} knowing many of them. I studied LSL during my work on \textbf{Second Life}: in fact I have worked on many Italian metaverse project such as Assis \url{http://www.secundavita.it}, Milan and Marostica on the \textbf{''Italia Vera''} project.\newline
I have a good knowledge of graphic applications (Gimp, Photoshop) and office tools such as Openoffice.org and iWork (for OSX)\newline
Determinated, I can work both in a team and individually and I am able to manage independently my workflow. Working usually in a team, I have a constructive and collaborative relationship with the people around me, such as colleagues and collaborators. I am sociable, friendly and with good communication skills; my site is a source of continuing social contact and exchange with other technical and less technical people.}


\end{document}
